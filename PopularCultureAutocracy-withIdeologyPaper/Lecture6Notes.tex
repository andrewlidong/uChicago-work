\documentclass{article}
\usepackage{graphicx}

\begin{document}

\begin{titlepage}
	\centering
	{\scshape\LARGE University of Chicago \par}
	\vspace{1cm}
	{\scshape\Large Popular Culture and Autocratic Governments\par}
	\vspace{1.5cm}
	{\huge\bfseries Week 6 Seminar: Performing under Autocracy $|$ Syrian Ambiguities \par}
	\vspace{1cm}
	{\Large\itshape Andrew Dong\par}
	\vspace{2cm}
	
	\vfill
\begin{abstract}
Week 6 (November 4): Performing under Autocracy—Syrian Ambiguities
 Stars in Broad Daylight, film by Ossama Mohammed

Readings:
Wedeen, Lisa. Ambiguities of Domination: Politics, Rhetoric, and Symbols in Contemporary Syria. University of Chicago Press, 1999 [New Edition 2015] 
 

\end{abstract}

\vfill

% Bottom of the page
	Professor~Lisa \textsc{Wedeen}
	\vspace{5 mm}
	\\{\large \today\par}
\end{titlepage}

\title{Popular Culture Art and Autocracy Week 6}
\author{Andrew Dong}

\maketitle

\begin{abstract}
These are notes from the 6th seminar of Popular Culture Art and Autocracy\end{abstract}

\section{Introduction}
We watch a clip about Usama Muhammad.  Stars in Broad Daylight.  Two children are ordinary villagers and they say things they probably don't understand but are related to Arab uprising.  


\subsection{Longer Clip from Stars in Broad Daylight}

A double wedding in a small village turns to high drama when one bride runs away and the other refuses to go on with her marriage. The drama unveils the fragile balance holding together a family strained by an abusive father now replaced by the successful but corrupt eldest son, a pathologically enraged second son, and the troubles of the youngest son, rendered deaf by a violent blow his father dealt him as a child. Ultimately tragic, the film is rife with biting humor and sharp political critique as it exposes how the violence of arbitrary and absolute power in a patriarchal society seeps into the unit of a family. Stars in Broad Daylight, Usama Mohammad’s first long feature, remains banned from screening in Syria because of its subversive representation and critical voice. Selected at the Quinzaine des Réalisateurs at the Cannes Film Festival in 1988.Filmmaker’s BiographyBorn in Lattakiya in 1954, Oussama Mohammad graduated from the Russian State Institute of Cinematography (VGIK) in 1979. There, he directed a short documentary, titled Khutwa Khutwa (Step by Step, 1978). He returned to Syria and directed a short documentary for the General Organization for Cinema titled Al-Yaom Koll Yaom (Today Everyday, 1980). He worked as assistant director to Mohammad Malas on Ahlam al-Madina (Dreams of the City, 1983) and directed his first fiction feature Nujum al-Nahar (Stars in Broad Daylight) in 1988. Deemed by many to be the most scathing critique of contemporary Syrian society trapped in the iron grip of the Baath regime, the film has never been allowed a public screening in Syria. Although not officially banned, the film has been shelved by diktat, and sits in storage under threat of irremediable physical deterioration. The film was selected at the Cannes Film Festival’s Quinzaine des Réalisateurs, and earned the filmmaker great critical praise, including the Golden Olive at the Valencia Festival in the same year. 

\vspace{5mm}

In 1992, he co-authored the script for al-Leyl (The Night, 1992) with Mohammad Malas and co-directed with Omar Amiralay and Malas the documentaries Shadows and Light (1991) and Fateh Moudaress (1994). He was unable to make his second feature until 2002. Sunduq al-Dunya (Sacrifices, 2002) was meant as an hommage to Andreï Tarkovsky’s The Sacrifice, the exiled Soviet master’s last film, and was selected for the Cannes Film Festival’s section Un Certain Regard in 2002. Complex and visually stunning, the film has confirmed its maker as one of the Soviet film school’s graduates most individual and masterful filmmakers.


\vspace{5mm}


\subsection{Thoughts on Stars in Broad Daylight}

\vspace{5mm}

Nalepa asks 3 questions to clarify.  The recitation by the twins in the beginning, was that sheer parody or does that actually happen?  It could have juts been extreme preemption where somebody was just trying to be oversupportive of the regime by teaching kids who had no idea what was going on.  

\vspace{5mm}

Nope, this is totally ordinary.  In the film's context it is parodying itself.  

\vspace{5mm}

There are two weddings that happen.  These children are being enjoined to recite bathist slogans and that is not being done at least in the understanding of the film maker and the structure of the film.  

\vspace{5mm}

The way it's set up you can tell in the form that the film is taking that it is both a criticism of the situation.  It is not happening as a talisman to ward off evil because these people are the regime.  

\vspace{5mm}

What was the purpose of replacing the president's picture with the iconic fictional character?  

\vspace{5mm}

He wanted to make sure that he could film that iconic hedeographic representation of a male figure.  It would have been very difficult to do that of the president himself without producing a kind of criticism that would have allowed him to stay safe.  It was only Asset's face and some iconography of the family and a few of combatants.  Andy Warhol like in the sense that it was the representation of Asset's face, less of other people because they could always offer a potential critique or substitution or suggest a type of replacement.  For the most part it was the political family being represented.  Because he was at least in the official rhetoric omnipresent.  

\vspace{5mm}

Nalepa: Clearly the audience were Syrians.  How does the artist see his role in an authoritarian regime.  Is he a dissident, is he somebody who enables communication?  

\vspace{5mm}

Ossamma Muhammed was probably on the far end of the artistic spectrum in terms of dissidence.  It was allowed to be produced, it was produced under the auspices of the Syrian film syndicate.  If you weren't produced by them there was no possiblity of filming at that time.  It wasn't allowed to be distributed.  This is where they gave a bit of latitude to artists without the pleasure of actual audience.  Because it was never distributed it was shown at more private events.  Over time more often.  You might be able to see it in a stadium outside but not in 1988, 1989.  For many of these films you wouldn't catch onto if you weren't Syrian did not mean that other people weren't watching it.  The main viewers, who would be able to come say to these quasi private showings or public showings that were un commercialized, film festival things largely (though some were in the country) (there were many opportunities for him to show the film), also it was shown overseas and received special recognition at Cannes.  There was enough that was deemed artistically interesting about the film and an auteur sensibility (mirrors, eggs) that made it the recipient of several awards.  Venice and Carthage.  Many of these directors ended up having careers that were just as important global as they were local.  Very little of these films had a commercial release available to them.  

\vspace{5mm}

To what extent was there an sort of interprative dialogue between centers and committees and the artists because it seems to walk that line you have to have a pretty solid grasp of these sort of subtleties.  Government surveilance of Langston Hughes, these government agencies would closely follow these artists.  

\vspace{5mm}

How censorship worked at the time in Syria.  This is 1988 and the massacre at Hama had happened in 1982.  To refer to the president in any way that wasn't submersive was trouble.  So, you submit a script and don't describe how the main characters look.  You keep your script to a minimum and you don't do a dummy script but you do something not so far away from it.  Just a script, not a shooting script.  There's a lot of room for extemporaneous, improvisational moments and that' strue in the 2000 period and you see comedies at work.  What the censors are approving is not what is occurring on the screen and part of that is subterfuge.  

\vspace{5mm}

What's also true is that no Syrian censor would think thta this was in any way an apolitical film because of the way in which the countryside is being depicted, the accents of the people, the sense of country bumpkinness being tethered to a system of regime.  The rape scene, the crassness of those people who are regime people, the use of the blanket with the lion on it are all too overt.  

\vspace{5mm}

Nalepa asks if the symbols are ambiguous.  Wedeen disagrees, they allow for a kind of disavowal.  It's true that many intellectuals ended up seeing this film.  Usama Muhammad wanted to do something that was true to him and reflected a fictionalized version of something that had become unbearably corrupt, and wanted a critical acclaim from an international audience.  

\vspace{5mm}

There is a structuralism to Usama Muhammad.  Birth and death, darkness and light, urban and rural, that would seem to be kind of orienting.  New self, old self.  There's enough mixing that you often then in some ways have those polarities but nowhere to stand.  

\vspace{5mm}

No one is left innocent here.  This is not a Mr. Smith goes to washington story.  (look this up).  The move to Damascus is a play from a move from the Hinterland into the heart of the capital city where decisions are made.  It's not innocence that you're bringing to that city it is already a corrupt and damaged being.  

\vspace{5mm}

Scene with the kissing and the hugging.  The minute you're disobeyed, slapping.  There's a lot of excess and that excess isn't simply in terms of the ways that people speak but also with what they do with their bodies.  It can be violent or it can be excess affection.  

\vspace{5mm}

These worlds are more comfortable with homosexuality than other parts of the world, such as the US might be.  There's a security that both use with a lion on it and there is a male reliance on one another in that way.  There may be something more there though.  Worth looking into.  

\vspace{5mm}

To go back to the point on the mentally challenged brother and the twins, what's so uncomfortable about them is they are so likable yet they are so moldable.  You feel sympathy for them but they are upholding the regime.  

\vspace{5mm}

In addition to the violence there is a great deal of tenderness, and in additional to the tenderness a great deal of rote speak.  

\vspace{5mm}

Other thoughts? Ok.   Quick break.  Another film also by Usama Muhammad, actually his first film.  Actually something he did when he was at Moscow university and has a new light in the context of the uprising because it is a quasi documentary.  After that film we'll talk about some of the other arguments in the books.  

\vspace{5mm}

\subsection{Hama Massacre}

The Hama massacre (Arabic: مجزرة حماة‎) occurred in February 1982, when the Syrian Arab Army and the Defense Companies, under the orders of the country's president Hafez al-Assad, besieged the town of Hama for 27 days in order to quell an uprising by the Muslim Brotherhood against al-Assad's government.[2][3] The massacre, carried out by the Syrian Army under commanding General Rifaat al-Assad, effectively ended the campaign begun in 1976 by Sunni Muslim groups, including the Muslim Brotherhood, against the government.

\vspace{5mm}

Initial diplomatic reports from Western countries stated that 1,000 were killed.[5][6] Subsequent estimates vary, with the lower estimates claiming that at least 10,000 Syrian citizens were killed,[1] while others put the number at 20,000 (Robert Fisk),[2] or 40,000 (Syrian Human Rights Committee).[3][4] About 1,000 Syrian soldiers were killed during the operation and large parts of the old city were destroyed. Alongside such events as Black September in Jordan,[7] the attack has been described as one of "the single deadliest acts by any Arab government against its own people in the modern Middle East".[8] According to anti Syrian government claims the vast majority of the victims were civilians.[9]

\vspace{5mm}

According to Syrian media, anti-government rebels initiated the fighting, who "pounced on our comrades while sleeping in their homes and killed whomever they could kill of women and children, mutilating the bodies of the martyrs in the streets, driven, like mad dogs, by their black hatred." Security forces then "rose to confront these crimes" and "taught the murderers a lesson that has snuffed out their breath".[10]

\subsection{Step By Step}

Never shown.  Deemed too potentially inflammatory, too critical. We want to think about why that would be the case.  

\vspace{5mm}

Thoughts?  

\vspace{5mm}

A bunch of shots that just go through a day in Soviet Russia from the beginning to the end.  Consider the scene where the kids are sleeping.  Very much a movie about the every day, though it's pretty clear that it's not one day.  Very much a kind of meditation on the ordinary, the ordinary in this particular tobacco growing village.  

\vspace{5mm}

Stars in Broad Daylight seemed more politically critical since Step by Step more so emphasizes the impoverished conditions.  Stars in Broad Daylight had more cues and metaphors.  

\vspace{5mm}

There's a juxtaposition of the ordinary mundane and you could hear in the background Happy May Day, perhaps critiquing the closing of a social contract as if they were supposed to take care of them.  They're impoverished and they're working on May Day.  Discourse celebrating the workforce and exploitative conditions where actual workers are being compelled on their holiday to work.  

\vspace{5mm}

Specifically what the loud speaker is saying is discussing the worker's sacrifice, and there's the juxtaposition of the worker being slapped.  To whom have the workers sacrificed and for what.  The same thing is repeated on the level of the military.  You have to some extent a sociological account of why you would go into the military.  There's no exit otherwise.  Life is too slow, the teacher's are too cruel, somehow that channels you into a military world where that violence is somehow acceptable.  

\vspace{5mm}

One of the jokes in the book about carrying the leader across the river, I'm not going to dive into the river because I have family and that's why I'm going to cross.  Painted an image of what kind of sacrifice you're willing to make for your leader.  

\vspace{5mm}

There's a big a shift between the seventies here and what overtime as the Bachist party loses it's claims to charisma and isn't simply routenized but corrupted.  Rhetoric that did matter to people in many ways became tired and palpably unable to generate the same level of energy.  Time really matters here.  It's at the height of a certain kind of developmentalist imaginary.  

\vspace{5mm}

Second thing to recognize is that this film is about the regime's hinterland, the area where they galvanized the most support.  That world is sort of off limits to folks in urban areas and people who are not of the autoleex.  Seeing the film 30 years later, because it was such a fine-grained documentary of the ordinary of a world that has for the most part been off-limits to them.  

\vspace{5mm}

The book has a weakness in that it doesn't take into account enough the way in which this rhetoric works for this small group of loyalists.  In fairness though, Professor Wedeen was looking at a very narrow rhetoric herself.  

\vspace{5mm}

There's a tenderness in the filmmaker in his ability to capture that childhood innocence that's trying to capture the hope that's been already lost in the older generations.  

\vspace{5mm}

Really drives it home these promises of upward social mobility seem to be unreachable just like a plane that's going overhead to symbolize something that's wholly unreachable.  

\vspace{5mm}

Documentary filmmaking had not been well developed and was not supported by the Syrian film syndicate.  There was one documentary filmmaker in Syria who managed outside funding.  Documentary film mostly not allowed, it had to be fictionalized.  There was an interest in filming the facts to make the truth vulnerable to ... ... uncertainties reign supreme.  The regime was able to win the media war simply by not losing it.  Sewing doubt was part of the regime's strategy.  Citizen journalists would go into the street and there became more and more interest in the documentary as a way of registering the truth.  

\vspace{5mm}

The film revealed a world that people were quite curious about but hadn't known.  Part of that world was a pretty rural part of La Die.  The world of the military was part of this as well.  It was revelatory of that world and of a way of filmmaking that seemed for some folks who were not necessarily all that versed in film or film theory to imagine that filming something that was happening was to in fact document the truth.  

\vspace{5mm}

Nalepa has a question about the quasi-documentary aspect of it.  The first part shows the idyllic cyclic nature of the country life and that's broken by the airplane.  The only person who actually is talking is the woman who in the first part ... triumph of the will, blending party religion state, he's the same person ... it's bookended by a kind of homage to the leader on the one hand and an homage to the party on the other hand and he's a kind of village sheik type person.  He has a type of religious authority but it's not a particularly learned take.  He is somebody who is a patriarch of the family.  He has the ability to convey the secrets of this religion to male progeny.  

\vspace{5mm}

There are also moments of revelation for people inside that community but also just the way people in that community interact.  The quasi documentary is in part because it's as if it seems chronological but it's not.  There are other ways too and the airplane is one of them.  there was no airplane, that is completely aspirational.  The lofty heights of the airplane are also symbolizing that kind of aspiration.  

\vspace{5mm}

Imagining a kind of future that is not so grounded in the future but that is literally aloft, but also takes you to places that are outside of Syria.  

\vspace{5mm}

Going back to the quasi documentary aspect, one thing that he does when he's interviewing certain people you have the older man who's normally in the house, and you're interviewing the man who's up on that hill.  Even when they're in the city they're bringing up these materials to build stuff for people who aren't them.  Their labor is embodied in the elevation of others.  

\vspace{5mm}

Most of the community until the late 70s was extraordinarily poor and one of the things that happened when this party came to power was the enabling of the provision of goods and services to the countryside.  About 3 percent of Syria was electrified and by 2000 it was 98 percent.  In 30 years you had the entire electrification of the country.  

\vspace{5mm}

Even a decade later that was going away.  One thing this state did was to protect and enable minorities largely through the armed forces.  What protection meant was slightly different but there was a kind of fantasy and a set of guarantees for a kind of multisectarian accomodation that this regime was able to secure.  In that context it's not true that all ollerbees were pro-regime.  There are religious distinctions among them.  There are political disagreements, whether you saw yourself as part of a truncated, a united Syria or a arab state.  Bathist party was more fascist.  Syrian socialist national party was more fascist still.  All had projects for the redistribution of income.  None of them were mass mobilization parties like the Nazi party or even the aspirations of the Bolshevik party in the Soviet Union.  Never had that kind of mobilizational uptake, which is why the discourse ends up fraying in the way it does.  When Wedeen arrives in the late 80s she was already seeing a palpable ambiguity that defines the book.  

\vspace{5mm}

\subsection{Lisa Wedeen's Experiences in Syria}

Started going to Syria in 1985.  Got interested in Syria almost accidentally but not obviously.  Hated the united states government.  Won two paper awards as an undergrad and wrote on the UAR united arab republic which was an alliance between Syria and Egypt.  The other paper was on Machiavelli.  Pooled money to get a ticket to anywhere in the world.  Decided to go to a place that was coated as the axis of evil.  At the time it was North Korea, Syria or Iran.  

\vspace{5mm}

North Korea was out, the idea of going was scary.  Probably wouldn't have gotten a visa . Iran interested her, but if she didn't like it would only be learning farsi.  Syria had more opportunities.  

\vspace{5mm}

Bought the ticket and arrives in Damascus.  Gets on the plane and asks for a hotel.  Asks the cab driver to take her to a hotel.  Pretty shitty hotel.  haha this is great.  Presumably it's a hotel for prostitutes, there was a luxury hotel across the street, and maybe this hotel serviced that hotel.  Reading Gabriel Garcia Marquez 100 years of solitude.  First day just stayed in her room and finished the book.  

\vspace{5mm}

Eventually went outside and asked the hotel if they wanted an aerobics teacher in exchange for a good shower.  Taught aerobics at the fancy hotel across the street.  It was to a group of primarily but not exclusively Christian families who were very well off who had health spa passes but had never had an aerobics teacher.  Introduced aerobics to Syria.  Haha so awesome.  

\vspace{5mm}

Through that met a lot of elites.  One family asked if she wanted to come live with them in exchange for babysitting their kids.  Lived at this place for 3 months, took arabic, fell in love with the place.  

\vspace{5mm}

1989 moved back into the dorms and was in her 4th year of arabic.  Loved Syrian dialect.  Lived in the dorms, they cost 7 dollars and 50 cents for the year.  Never saw running water in the dorms.  Had a roommate and lived over the cafeteria and the meat smell would waft into her room.  Met a tremendous number of interesting people in the dorms, primarily women who didn't have homes.  

\vspace{5mm}

Lived in the dorms for a year, again taught aerobics in exchange for running water and met a lot more people, some of whom remain friendly.  Best friend from that time period is now living in Chicago, so that's 24 years of friendship.  Little by little started to go back.  

\vspace{5mm}

First writing the book, she was going to write a dissertation book on how it was that these symbols produced legitimacy.  Then she got to Syria and realized that by 1988, 1989 but certainly 1992.  

\vspace{5mm}

Talked about issues from Nietzsche to jokes till 3 in the morning, and they could stay but then had to leave at 10.  Basically happened for a year.  Initially came up with the project of figuring out how symbols worked to produce legitimacy.  

\vspace{5mm}

Sometimes legitimacy means the moral right to rule.  Sometimes legitimacy means popularity.  That raises epistemological and metaphysical questions, how do you know what people mean.  

\vspace{5mm}

Little by little we figured that the symbols weren't doing the work that rhetoric and symbols do.  Nor could it be dismissed that politics were about political interests and what the groups suggested about them.  The regime was spending scarce resources on this kind of activity as opposed to more carrots and sticks.  As opposed to more ways of punishing people or actually offering inducements for compliance.  So what work were these rhetoric and symbols doing?  Particularly because they couldn't be said to do any legitimacy work because legitimacy presupposes work and belief.  These were undeniably generating more unbelief than belief.  No way to empirically show this, but nor could they be dismissed as something that was simply empaphenomenal.  

\vspace{5mm}

Argued that they mattered in a kind of disciplinary work, to reduce complicity, atomize people from one another.  They took things that were extraordinarily important to folks in the 1970s and made them seem silly and absurd.  They nonetheless helped to keep people safe because they produced the guidelines for a form of acceptable behavior.  Far from being disorienting, by the late 80s and 90s, these were a way to stay safe.  Allowed one to be orienting to the world.  

\vspace{5mm}

Had to be able to show that people didn't believe these spurious slogans.  Began to collect evidence but that wasn't the type of evidence that political scientists generally collect, because the domain wasn't one that scientists normally paid attention to.  

\vspace{5mm}

That's how we get to chapter 4, which looked at collected jokes but also looked at more blatant kinds of transgresent behaviors like the cartoons, like the kinds of television serials, and that prompted me to ask a further question.  What work were these artistic transgressions doing?  

\vspace{5mm}

There we use people like Slavoj Zizek and other eastern european philosophers.  These modes of transgression can be seen as political resistance to the extent that they undermine the isolating aspects of the cult by demonstrating that one's unbelief is collectively shared.  

\vspace{5mm}

On the other hand, however, they also shore up another aspect of the cult.  They undermine it in one way and shore it up in another.  How?  Consider Zizek.  

\vspace{5mm}

They demonstrate unbelief and that unbelief is important for generating a sense of solidarity on the one hand but on the other hand that shared unbelief is one of the ways that the cult reproduces itself.  Because the cult generates it's political power through obediance, and obediance as opposed to good culture relies on not believing.  Forces one to not believe, and shows that everyone is complying.  Complying is what ultimately matters here.  

\vspace{5mm}

!!!!!

\vspace{5mm}

Nalepa: 
What makes resistance in a cult so difficult is that the regime gets citizens to realize how far away their private lives.  There's such a huge dissidence between what they personally believe and what they actually do.  It's almost uncomfortable to switch to the other side.  

\vspace{5mm}

In the process of writing this, did anybody push you in the direction of this just being Kuran.  

\vspace{5mm}

When she started doing this Kuran wasn't published yet.  When she was moving towards a publication, UChicago tried pushing her into a Kuran story.  Why does it not work?  

\vspace{5mm}

That division between public and private is simply not as stark as what Kuran wants it to be.  This resistance was also publically shared.  

\vspace{5mm}

The idea that your private junctures are so private actually was not empirically true, and then that empirical problem allowed me to rethink the theoretical story.  

\vspace{5mm}

Similarly, the Jim Scott story in many ways does a very good job of describing what happens on and off stage.  Much of what she was talking about was publically available to everyone.  People were in fact working the weaknesses of this.  

\vspace{5mm}

Doing a kind of salutary work of the regime because the regime didn't require belief in order to be powerful.  Would it have been better off it could win the hearts and minds of the citizens?  Sure.  But it didn't matter.  

\vspace{5mm}

In some ways, Wedeen as Kuran does relies heavily on Eastern European philosophers and thinkers because their experience was the closest thing to what she was also witnessing.  Witnessing it in a way that was ultimately closer to Zizek than it was to Havel.  

\vspace{5mm}

Aspects of Havel did actually matter to her.  Mark Hansen in chapter 3 allows Wedeen to say that truth doesn't matter, perception of truth is much more significant.  People would pick it up because it resonated with them.  

\vspace{5mm}

Foucault power relations and for every power relation there is a form of resistance.  Forms of resistance encountered in the Syrian regime done on the part of the citizens.  Have you encountered mechanisms that do a different kind of work that is not affirmative of the regime?  

\vspace{5mm}

For sure.  Most of the uprising was not.  Acting as if is not the same thing as having a huge necessarily disjuncture between public and private because the acting as if is widely enough shared in public to be known that you do not believe.  

\vspace{5mm}

What makes laughter so important?  In part it's a reverance to the status quo.  There's a kind of infectiousness that has to do with it being shared.  Even in places like homes these were private enclaves of publicity.  

\vspace{5mm}

Wedeen a student of Foucault's.  At the University of California Berkeley as an undergrad and Foucault transformed the way she thought of the world.  There are limitations though, namely in the sense that you are always shoring up the very thing you wish to repudiate and disavow.  That's a very cynical almost functionalist account.  Foucaultians tend to have that refrain.  

\vspace{5mm}

Wedeen wanted to gain traction and insights from Foucault but not necessarily take on that kind of automatic, taken for granted refrain.  There are forms of refrain.  Most forms of resistance actually stretch conventions but don't in fact challenge them to the point of breaking.  

\vspace{5mm}

When the uprising started, living in Syria, and the book was selling very well, one of the things that was true about it was that the book took on a life of its own.  The Foucaultian story is quite dark, whereas people who were reading it were seeing this as a kind of recognition of that counter force that had been held in reserve.  They were using it as a kind of affirmative text for revolution.  And that's how it was being reread.  And that surprised her.  

\vspace{5mm}

Many arguments with people, though Wedeen became a bit of a celebrity.  The book suggests that these kinds of power relations were very hard to undo.  In a regime that has the power to crush things whenever they fail, you need to be prepared for this.  

\vspace{5mm}

Wedeen sees political struggle as being extraordinarily difficult.  Autocratic regimes economize on coercive force, but that coercive force doesn't drop out of the picture.  It's always in reserve.  

\vspace{5mm}

Nalepa asks, Are these takes premeditated on the part of the regime?  No, probably not deliberate but work with a type of momentum.  

\vspace{5mm}

When you do field work and you realize you have no project, Wedeen crawls back into her funny dorm room.  Wedeen recommends bring lots of books.  Bring books and read promiscuously.  Read outside of the narrow terrain that is your subject.  

\vspace{5mm}

Reading a book called Renaissance Self Fashioning.  This story in there of Cardinal Wolzy, a major figure in Tudor England under Henry the 8th.  Cardinal Wolzy had just delivered a speech and it's partially cited in Chapter 8.  Giving a speech and asks his colleagues what they think.  Mellifluous prose.  Everyone's upping rhetorical ante, but there's only so much of this that you can do.  Sire I'm speechless.  

\vspace{5mm}

Greenblatts point wasn't that this was fictional, but that you could enforce that fiction.  

\vspace{5mm}

That disappointment turned into a new question.  Her book!  

\vspace{5mm}

In light of the recent events in Syria, how has your view on Syria changed?  

\vspace{5mm}

In the context of this uprising, one thing that dissapointed a lot of people is that Wedeen is not an intellectual.  She doesn't think it is her job to be an activist.  Doesn't mean she doesn't have a politics, but she thinks tha intellectuals need to have a sort of estrangement from the situation.  Needs to continue writing in a situation that tends to be unbearable for most activists.  Which isn't to say that she doesn't have sympathies.  Against torture, dictatorships, the ways corruption have contaminated a resistance that had potential.  Interested in thinking about forms of language that might convey the tenderness that she feels for a variety of fields, but that's not activism.  


\section{Conclusion}
Great class.  


\end{document}