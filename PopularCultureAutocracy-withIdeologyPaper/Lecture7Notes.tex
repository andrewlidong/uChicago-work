\documentclass{article}
\usepackage{graphicx}

\begin{document}

\begin{titlepage}
	\centering
	{\scshape\LARGE University of Chicago \par}
	\vspace{1cm}
	{\scshape\Large Popular Culture and Autocratic Governments\par}
	\vspace{1.5cm}
	{\huge\bfseries Week 7 Seminar: Soviet Paradoxes \par}
	\vspace{1cm}
	{\Large\itshape Andrew Dong\par}
	\vspace{2cm}
	
	\vfill


\begin{abstract}
Week 7 (November 11): Soviet Paradoxes
Readings:
Yurchak, Alexei. Everything Was Forever, Until It Was No More: The Last Soviet Generation. Princeton University Press, 2013 (Chapters 1–5) 
\end{abstract}

\vfill

% Bottom of the page
	Professor~Lisa \textsc{Wedeen}
	\vspace{5 mm}
	\\{\large \today\par}
\end{titlepage}

\title{Popular Culture Art and Autocracy Week 7}
\author{Andrew Dong}

\maketitle


\section{Introduction}

We do some reading from \textit{Everything was forever until it was no more}.  

\vspace{3mm}

Nalepa: In the Soviet context, who plays the role of this master?  Stalin.  In what ways does he play that role?  

\vspace{3mm}

Omar: He ends all discussion and interpretation that was going on before his time and becomes the ultimate source of wisdom and becomes uncontestable.  

\vspace{3mm}

Florence: Builds Soviet institutions that provide data and knowledge in a very Soviet mindset.  Before he came to power the research was done by elites from the former empires.  

\vspace{3mm}

His top-down approach is very repressive but a lot of it is focused on controlling in a way that most people have some kind of freedom.  

\vspace{3mm}

He edited texts and speeches.  Cut out encyclopaedia entries.  

\vspace{3mm}

Nalepa used to teach East European politics and got the Polish constitution of 1952 with Stalin's personal annotations.  Were you as stunned as Nalepa was about how directly he would engage with readers who would ask questions about readers? 

\vspace{3mm}

People living in Poland have grown to portray Stalin as a terror figure.  Then he dies, Krustchev comes to power and what happens to the processes suggested?  

\vspace{3mm}

There's destalinization within Russia as a result of Kruschev, and as a result of that authority of discourse becomes the master.  This leads to hyper-normalization of the discourse.  

\vspace{3mm}

This development of this official language, does it seem completely alien to you?  

\vspace{3mm}

I don't really know what they're saying here.  

\section{Everything was Forever, Until it was No More}

Soviet socialism was based on paradoxes that were revealed by the peculiar experience of its collapse. To the people who lived in that system the collapse seemed both completely unexpected and completely unsurprising. At the moment of collapse it suddenly became obvious that Soviet life had always seemed simultaneously eternal and stagnating, vigorous and ailing, bleak and full of promise. Although these characteristics may appear mutually exclusive, in fact they were mutually constitutive. This book explores the paradoxes of Soviet life during the period of "late socialism" (1960s-1980s) through the eyes of the last Soviet generation.


Focusing on the major transformation of the 1950s at the level of discourse, ideology, language, and ritual, Alexei Yurchak traces the emergence of multiple unanticipated meanings, communities, relations, ideals, and pursuits that this transformation subsequently enabled. His historical, anthropological, and linguistic analysis draws on rich ethnographic material from Late Socialism and the post-Soviet period.


The model of Soviet socialism that emerges provides an alternative to binary accounts that describe that system as a dichotomy of official culture and unofficial culture, the state and the people, public self and private self, truth and lie--and ignore the crucial fact that, for many Soviet citizens, the fundamental values, ideals, and realities of socialism were genuinely important, although they routinely transgressed and reinterpreted the norms and rules of the socialist state.

\vspace{3mm}

Stalin did read Plato but the Soviet union's efforts were to make it into a mass party.  With Plato there are these philosopher kings and they are the ones with access.  

\vspace{3mm}

The art world among other worlds needed to be also adjudicated in this way is quite important, not by the market but by people who had the right political judgements.  In part that was because everything was sponsored by the state.  Every official art production had this accompanying either party cell.  These people had to spend their hours on something.  Then Marxist Leninist philosophy was attached to the curriculum of grade school, secondary school through college.  

\vspace{3mm}

Once Kruschev denounced the cult of personality it was not going to be organized in the same way (socialism).  We see this in the exercises of speech writing where the author is further and further removed.  Reporting the news in such a way that doesn't make it news anymore.  This period of late socialism and extreme rigidity and cementing of soviet conventions.  The contribution of Yurchak's book is the demistification of the idea that after this rigidity there was nothing.  

\vspace{3mm}

Shows how this last socialist generation was actually incredibly culturally rich.  

\vspace{3mm}

Why wasn't there another figure like Stalin?  When we read Svalik's book chapter 3, one of the options of the game-theoretic analyses.  Powersharing with the elite was much easier.  

\vspace{3mm}

Stalin's cult of personality was already a second generation cult because there was the Leninist cult earlier.  The soviet union already had a cult and this wasn't undermined.  Stalin's cult of personality is one of the longest lasting ever.  

\vspace{3mm}

The Lenin cult of personality when he was alive was very different from the cult of personality when he died.  It was almost idolatry.  

\vspace{3mm}

Stalin dies, but the two questions are why didn't the soviet leadership choose to perpetuate a cult of personality and the other aspect of that question is why did they do what they did instead?  These questions kind of come together in the following way. 

\vspace{3mm}

This is an argument that Ken Jarrett makes in the Leninist extinction.  As opposed to say a fascist regime in which the individual leader is endowed with that supernatural capacity, in the context of Bolshevik regimes it's the party that has power.  It did not have to rely on the leader for some of the reasons that Kruschev brought to the floor.  

\vspace{3mm}

There's a routinization of that charisma.  Charisma has a transformative moment.  Then it has a consolidation moment and that's Stalin.  And then the routinization.  

\vspace{3mm}

It's worthwhile talking about what kind of party the communist party was.  The communist party was the avant-garde of the labor class.  In Leninism it was supposed to be temporary.  But one of the things that occurred during Stalin's reign was that the revolution was not spreading to the rest of Europe.  It had started and stopped in countries that were least likely to experience a revolution.  Countries with enormous peasant populations.  That part of Leninist philosophy that was connected to the party's temporary aspect were no longer valid.  The revolution is not spreading and the party is there to stay.  How could socialism guarantee employment?  There was this double layer of administration.  It's not the kind of party that we come to think of but some aspects of it are in common, for instance charisma and party are very difficult to combine in one.  

\vspace{3mm}

It helped Stalin shift that charismatic authority away from the party towards himself in the context of world war 2.  That may have made it easier to happen.  Struck by how extraordinarily hands on and knowledgeable he was.  The fact that he could intervene on policy.  There was a kind of level of education and expectation of a ruler that was important to the maintenance of a kind of party authority.  

\vspace{3mm}

Bolshevik regimes are learning from each other in a variety of ways, there's a lot of cross polination of ideas.  

\vspace{3mm}

World War 2 in Russia is called the Patriotic war.  Identified the purpose of Russia as combating fascism.  

\vspace{3mm}

Stalin's lingering reception in places like Poland.  How is Stalin received or understood in places of the former bloc.  Most of those countries in the Soviet Bloc had their own Stalins.  Poland's Stalin figure was Guierro.  The focus on condemning the Stalinist period is condemning the Guierro period.  This is when most political prisoners were sentenced to death.  

\vspace{3mm}

The focus on condemning the Stalin period is on that.  Nalepa's father remembers crying when Stalin died on the radio.  

\vspace{3mm}

Yurchak's argument about the way in which language in particular changes in the context of this shift away from Stalin towards the post-Stalinist regime.  He draws a lot on linguistic theory here.  He draws a lot on JL Austin in particular.  JL Austin is a theorist of ordinary language.  Very inspired by Wittgenstein in particular.  He makes the following argument: there are two major kinds of speech.  There's constative speech and there's performative speech.  Constative speech is speech that can be thought of in terms of true and false statements.  Austin became much more interested in the performative.  He discovered that in the first person singular active sense, verbs actually enacted the thing they named.  The statements themselves were deeds or actions.  I promise.  I warn you.  I hereby christen this ship.  Some of it was more contextual than others, but there were words that actually performed the action named.  I condemn.  I declare.  We hold these truths to be self-evident.  I do.  I assure you.  Performatives are such that Austin starts to discover the more interesting phenomenon that lots of verbs are performative in not that strict sense.  By performing the actions of patriotism I was thereby not a patriot.  Often actions would perform the context indexed.  Moreover, every moment of language was performative in even a looser sense.  Language is itself not only saying but also doing.  All language is performative.  There are 3 ways in which Austin could be understood to be thinking about performatives.  In performing an action I am performing the thing not necessarily named but indexed.  All speech is a form of action in it's own way.  

\vspace{3mm}

The declarations of objected truth that are rendered in those documents, those things stay the same and therefore they're getting kind of brittle.  But the people's version of the performative, their relying more on the performative, and their thinking their own thought.  The constative is little by little changing.  These things are constantly interacting with one another.  The truth statements no longer have the same sort of force that they had before.  But people can still resort to the kind of performative.  

\vspace{3mm}

Little by little the relationship between the performative and the constative begin to change.  The performance starts to feel more empty and more as an alibi to do whatever you want.  The constative statements therefore no longer have the same sort of validity that they had before.  

\vspace{3mm}

Political theorists pick up on this aspect in particular.  Yurchak is less interested in that than he is in sections 1 and 2.  What would be an example of this performative shift?  

\vspace{3mm}

The communist regime and religion were incompatible.  The communists tried to harness the power of this church by recruiting rank and file young clerics who they suspected would be critical of the episcopat and try to make them into these catholic communists.  One of the biggest problems that they faced was that these priests might be red but having gone through training they're being taught how to give sermons according to certain rules.  They had to teach these red priests how to build these new sermons.  Would publish these template sermons.  How, for instance, Christmas and the birth of christ could be connected with the birth of a new communist world.  Yurchak's interpretation would be that these red priests had to be retaught or acquainted with a new discourse.  

\subsection{Constative vs. Performative}

Workers have been historically exploited.  (constative statement because it is true or false and is a statement of description).  

\vspace{3mm}

What would a performative be in relation to this statement?  

\vspace{3mm}

Are constative and performative statements mutually disjoint?  According to Yurchak it's worthwhile to distinguish.  The statement may have aspects of both, but for Yurchak it's different domains of the same phenomena.  

\vspace{3mm}

The performative aspects of it would be a vote, yes they have or the activity of speech writing in which one has to take into account this language and maybe rework it.  Or the activity of listening to the statement while doing something else, or signing an oath I hereby declare that workers have been historically exploited.  

\vspace{3mm}

Yurchak trying to say that that's not something we want to read in either it's truth or its falsity but instead in terms of it's interactivity with the performative.  

\vspace{3mm}

Workers have historically been exploited.  I condemn the fact that workers have historically been exploited.  

\vspace{3mm}

The main problem for the new leaders was not to commit a political mistake by writing something irregular.  

\vspace{3mm}

Charisma is very spontaneous.  This is much more difficult when rule is depersonified.  There's no room for spontaneity once this deauthoritization happens.  

\subsection{Concluding Yurcan}

Most parties did not have a mass mobilization party in the same way that the Bolshevik, fascist parties did.  In part because the states were much weaker.  And you didn't have the same kind of ideological edifice that had a genealogy attached to it and lots of educated people thinking about it, signing up to be loyalists.  

\vspace{3mm}

Most parties never had the genealogical depth than the communists.  There was no Marx for the Bathists.  Bathist ideology was to some extent a hodge podge.  A bit of Marxism, a bit of Fascism, and a lot of nationalism that was more fascist than it was beholden to versions of socialism, but it didn't have very much depth or the same kind of uptake.  

\vspace{3mm}

How could there seem to be a party that seemed to be immutable and last forever seem to disappear?  

\vspace{3mm}

Having said that Wedeen will say that there is a slight misreading.  It's the one that Nalepa was asking about at the end of last time.  One reason why Kuran's work is problematic is because that binary is overdrawn in most of the work on the subject including Havel and Scott.  However, Wedeen does think that she resorts to that language.  Sometimes it's kind of true.  

\vspace{3mm}

There are also many enclaves of publicity - the ways in which the public and the opposition tolerated jokes in public.  Yurchak seems unfair by not looking at Chapter 4 in greater detail.  Things are hidden in public in plain view.  

\vspace{3mm}

Transgressive practices can be resistance on some levels and not on other levels.  

\vspace{3mm}

She will concede that there's something that she's getting at which has to do with this category of the svoy.  In most instances that's the way that people are living their lives.  On some level, we're looking for politics in everything.  The way that he describes things is also political but is not as dramatic as Wedeen is making everything else.  

\section{Svoy}

When we were reading Havel it was the green grocers who were upsetting the truth and who were doing something in their own interest.  It's the dissident's who are making it more difficult for everybody.  

\vspace{3mm}

Vneia is a category used towards dissidence.  

The Svoy is your circle of friends who you would be hanging out with who would be as political as you are.  These are the people who will sign your petitions, refuse to sign the ones you refuse to sign.  They'll be sharing your understanding of the rules.  

\vspace{3mm}

That sense of being inside and outside the system, Wedeen wonders if we could talk a little bit more about that.  It's a kind of estrangement where we're no longer being dissident.  

\vspace{3mm}




\section{Conclusion}

Concluding Thoughts

\end{document}