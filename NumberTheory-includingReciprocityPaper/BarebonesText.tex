\documentclass{article}
\usepackage{graphicx}

\begin{document}

\title{Popular Culture Art and Autocracy Week 6}
\author{Andrew Dong}

\maketitle

\begin{abstract}
These are notes from the 6th seminar of Popular Culture Art and Autocracy\end{abstract}

\section{Introduction}
We watch a clip about Usama Muhammad.  Stars in Broad Daylight.  Two children are ordinary villagers and they say things they probably don't understand but are related to Arab uprising.  


\subsection{Longer Clip from Stars in Broad Daylight}

A double wedding in a small village turns to high drama when one bride runs away and the other refuses to go on with her marriage. The drama unveils the fragile balance holding together a family strained by an abusive father now replaced by the successful but corrupt eldest son, a pathologically enraged second son, and the troubles of the youngest son, rendered deaf by a violent blow his father dealt him as a child. Ultimately tragic, the film is rife with biting humor and sharp political critique as it exposes how the violence of arbitrary and absolute power in a patriarchal society seeps into the unit of a family. Stars in Broad Daylight, Usama Mohammad’s first long feature, remains banned from screening in Syria because of its subversive representation and critical voice. Selected at the Quinzaine des Réalisateurs at the Cannes Film Festival in 1988.Filmmaker’s BiographyBorn in Lattakiya in 1954, Oussama Mohammad graduated from the Russian State Institute of Cinematography (VGIK) in 1979. There, he directed a short documentary, titled Khutwa Khutwa (Step by Step, 1978). He returned to Syria and directed a short documentary for the General Organization for Cinema titled Al-Yaom Koll Yaom (Today Everyday, 1980). He worked as assistant director to Mohammad Malas on Ahlam al-Madina (Dreams of the City, 1983) and directed his first fiction feature Nujum al-Nahar (Stars in Broad Daylight) in 1988. Deemed by many to be the most scathing critique of contemporary Syrian society trapped in the iron grip of the Baath regime, the film has never been allowed a public screening in Syria. Although not officially banned, the film has been shelved by diktat, and sits in storage under threat of irremediable physical deterioration. The film was selected at the Cannes Film Festival’s Quinzaine des Réalisateurs, and earned the filmmaker great critical praise, including the Golden Olive at the Valencia Festival in the same year. 

\vspace{5mm}

In 1992, he co-authored the script for al-Leyl (The Night, 1992) with Mohammad Malas and co-directed with Omar Amiralay and Malas the documentaries Shadows and Light (1991) and Fateh Moudaress (1994). He was unable to make his second feature until 2002. Sunduq al-Dunya (Sacrifices, 2002) was meant as an hommage to Andreï Tarkovsky’s The Sacrifice, the exiled Soviet master’s last film, and was selected for the Cannes Film Festival’s section Un Certain Regard in 2002. Complex and visually stunning, the film has confirmed its maker as one of the Soviet film school’s graduates most individual and masterful filmmakers.


\vspace{5mm}


\subsection{Thoughts on Stars in Broad Daylight}

\vspace{5mm}

Nalepa asks 3 questions to clarify.  The recitation by the twins in the beginning, was that sheer parody or does that actually happen?  It could have juts been extreme preemption where somebody was just trying to be oversupportive of the regime by teaching kids who had no idea what was going on.  

\vspace{5mm}

Nope, this is totally ordinary.  In the film's context it is parodying itself.  

\vspace{5mm}

There are two weddings that happen.  These children are being enjoined to recite bathist slogans and that is not being done at least in the understanding of the film maker and the structure of the film.  

\vspace{5mm}

The way it's set up you can tell in the form that the film is taking that it is both a criticism of the situation.  It is not happening as a talisman to ward off evil because these people are the regime.  

\vspace{5mm}

What was the purpose of replacing the president's picture with the iconic fictional character?  

\vspace{5mm}

He wanted to make sure that he could film that iconic hedeographic representation of a male figure.  It would have been very difficult to do that of the president himself without producing a kind of criticism that would have allowed him to stay safe.  It was only Asset's face and some iconography of the family and a few of combatants.  Andy Warhol like in the sense that it was the representation of Asset's face, less of other people because they could always offer a potential critique or substitution or suggest a type of replacement.  For the most part it was the political family being represented.  Because he was at least in the official rhetoric omnipresent.  

\vspace{5mm}

Nalepa: Clearly the audience were Syrians.  How does the artist see his role in an authoritarian regime.  Is he a dissident, is he somebody who enables communication?  

\vspace{5mm}

Ossamma Muhammed was probably on the far end of the artistic spectrum in terms of dissidence.  It was allowed to be produced, it was produced under the auspices of the Syrian film syndicate.  If you weren't produced by them there was no possiblity of filming at that time.  It wasn't allowed to be distributed.  This is where they gave a bit of latitude to artists without the pleasure of actual audience.  Because it was never distributed it was shown at more private events.  Over time more often.  You might be able to see it in a stadium outside but not in 1988, 1989.  For many of these films you wouldn't catch onto if you weren't Syrian did not mean that other people weren't watching it.  The main viewers, who would be able to come say to these quasi private showings or public showings that were un commercialized, film festival things largely (though some were in the country) (there were many opportunities for him to show the film), also it was shown overseas and received special recognition at Cannes.  There was enough that was deemed artistically interesting about the film and an auteur sensibility (mirrors, eggs) that made it the recipient of several awards.  Venice and Carthage.  Many of these directors ended up having careers that were just as important global as they were local.  Very little of these films had a commercial release available to them.  

\vspace{5mm}

To what extent was there an sort of interprative dialogue between centers and committees and the artists because it seems to walk that line you have to have a pretty solid grasp of these sort of subtleties.  Government surveilance of Langston Hughes, these government agencies would closely follow these artists.  

\vspace{5mm}

How censorship worked at the time in Syria.  This is 1988 and the massacre at Hama had happened in 1982.  To refer to the president in any way that wasn't submersive was trouble.  So, you submit a script and don't describe how the main characters look.  You keep your script to a minimum and you don't do a dummy script but you do something not so far away from it.  Just a script, not a shooting script.  There's a lot of room for extemporaneous, improvisational moments and that' strue in the 2000 period and you see comedies at work.  What the censors are approving is not what is occurring on the screen and part of that is subterfuge.  

\vspace{5mm}

What's also true is that no Syrian censor would think thta this was in any way an apolitical film because of the way in which the countryside is being depicted, the accents of the people, the sense of country bumpkinness being tethered to a system of regime.  The rape scene, the crassness of those people who are regime people, the use of the blanket with the lion on it are all too overt.  

\vspace{5mm}

Nalepa asks if the symbols are ambiguous.  Wedeen disagrees, they allow for a kind of disavowal.  It's true that many intellectuals ended up seeing this film.  Usama Muhammad wanted to do something that was true to him and reflected a fictionalized version of something that had become unbearably corrupt, and wanted a critical acclaim from an international audience.  

\vspace{5mm}

There is a structuralism to Usama Muhammad.  Birth and death, darkness and light, urban and rural, that would seem to be kind of orienting.  New self, old self.  There's enough mixing that you often then in some ways have those polarities but nowhere to stand.  

\vspace{5mm}

No one is left innocent here.  This is not a Mr. Smith goes to washington story.  (look this up).  The move to Damascus is a play from a move from the Hinterland into the heart of the capital city where decisions are made.  It's not innocence that you're bringing to that city it is already a corrupt and damaged being.  

\vspace{5mm}

Scene with the kissing and the hugging.  The minute you're disobeyed, slapping.  There's a lot of excess and that excess isn't simply in terms of the ways that people speak but also with what they do with their bodies.  It can be violent or it can be excess affection.  

\vspace{5mm}

These worlds are more comfortable with homosexuality than other parts of the world, such as the US might be.  There's a security that both use with a lion on it and there is a male reliance on one another in that way.  There may be something more there though.  Worth looking into.  

\vspace{5mm}

To go back to the point on the mentally challenged brother and the twins, what's so uncomfortable about them is they are so likable yet they are so moldable.  You feel sympathy for them but they are upholding the regime.  

\vspace{5mm}

In addition to the violence there is a great deal of tenderness, and in additional to the tenderness a great deal of rote speak.  

\vspace{5mm}

Other thoughts? Ok.   Quick break.  Another film also by Usama Muhammad, actually his first film.  Actually something he did when he was at Moscow university and has a new light in the context of the uprising because it is a quasi documentary.  After that film we'll talk about some of the other arguments in the books.  

\vspace{5mm}

\subsection{Hama Massacre}

The Hama massacre (Arabic: مجزرة حماة‎) occurred in February 1982, when the Syrian Arab Army and the Defense Companies, under the orders of the country's president Hafez al-Assad, besieged the town of Hama for 27 days in order to quell an uprising by the Muslim Brotherhood against al-Assad's government.[2][3] The massacre, carried out by the Syrian Army under commanding General Rifaat al-Assad, effectively ended the campaign begun in 1976 by Sunni Muslim groups, including the Muslim Brotherhood, against the government.

\vspace{5mm}

Initial diplomatic reports from Western countries stated that 1,000 were killed.[5][6] Subsequent estimates vary, with the lower estimates claiming that at least 10,000 Syrian citizens were killed,[1] while others put the number at 20,000 (Robert Fisk),[2] or 40,000 (Syrian Human Rights Committee).[3][4] About 1,000 Syrian soldiers were killed during the operation and large parts of the old city were destroyed. Alongside such events as Black September in Jordan,[7] the attack has been described as one of "the single deadliest acts by any Arab government against its own people in the modern Middle East".[8] According to anti Syrian government claims the vast majority of the victims were civilians.[9]

\vspace{5mm}

According to Syrian media, anti-government rebels initiated the fighting, who "pounced on our comrades while sleeping in their homes and killed whomever they could kill of women and children, mutilating the bodies of the martyrs in the streets, driven, like mad dogs, by their black hatred." Security forces then "rose to confront these crimes" and "taught the murderers a lesson that has snuffed out their breath".[10]

\subsection{•}
\section{Conclusion}
Write your conclusion here.

\end{document}