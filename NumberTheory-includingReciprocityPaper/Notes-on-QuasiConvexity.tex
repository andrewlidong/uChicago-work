 \documentclass{pset}

\newcommand{\vect}[1]{\vec{#1}}
\setlength{\parindent}{0pt} 
\setlength{\parskip}{2ex}
\usepackage{geometry}               
\geometry{letterpaper}                 
%\geometry{landscape}                
\usepackage[parfill]{parskip}   
\usepackage{graphicx}
\usepackage{hyperref}
%these aren't necessary because we're in an ams environment, but for completeness
\usepackage{amssymb}
\usepackage{amsfonts}
\usepackage{amsthm}
\usepackage{amsmath}
\usepackage{epstopdf}
\usepackage{graphicx}
\usepackage{enumerate}

%mathbb
\newcommand{\cof}{\mathrm{cof}}
\newcommand{\fp}{\mathbb{F}_p}
\newcommand{\fpn}{\mathbb{F}_{p^n}}
\newcommand{\N}{\mathbb N}
\newcommand{\Z}{\mathbb Z}
\newcommand{\C}{\mathbb C}
\newcommand{\Q}{\mathbb Q}
\newcommand{\R}{\mathbb R}
\newcommand{\E}{\mathbb E}
\newcommand{\F}{\mathbb F}
\newcommand{\Rn}{{\mathbb R}^n}
\newcommand{\Rm}{{\mathbb R}^m}
\newcommand{\al}{\alpha}
\newcommand{\cc}{\subset\subset}

%letters
\newcommand{\e}{\varepsilon}
\newcommand{\ph}{\varphi}
\newcommand{\Ph}{\varPhi}

\newcommand{\g}{\gamma}
\newcommand\limn{\lim\limits_{n \to \infty}}
\newcommand\limsupn{\limsup\limits_{n \to \infty}}
\newcommand\limt{\lim\limits_{t \to 0}}
%some text
\newcommand{\card}{\text{card}}
\newcommand{\image}{\text{image}}
\newcommand{\coker}{\text{coker}}
\providecommand{\abs}[1]{\left\lvert#1\right\rvert}
\providecommand{\norm}[1]{\left\lVert#1\right\rVert}
\renewcommand{\t}{\mathrm}
\newcommand{\Gal}{\mathrm{Gal}}
\newcommand{\Hom}{\mathrm{Hom}}
\newcommand{\Tor}{\mathrm{Tor}}
\newcommand{\rad}{\mathrm{rad}}
\newcommand{\End}{\mathrm{End}}
\newcommand{\Var}{\mathrm{Var}}
\newcommand{\spn}{\mathrm{span}}
\newcommand{\tab}{\hspace*{2em}}
%Various things you might want%


\theoremstyle{definition}\newtheorem*{defn}{Definition}
\theoremstyle{definition}\newtheorem*{eg}{Example}
\theoremstyle{theorem}\newtheorem{prop}{Proposition}
\theoremstyle{definition}\newtheorem{ex}{Exercise}
\theoremstyle{definition}\newtheorem*{question}{Question}
\theoremstyle{definition}\newtheorem*{answer}{Answer}

\theoremstyle{definition}\newtheorem*{oproblem}{Open Problem}
\theoremstyle{theorem}\newtheorem{thm}{Theorem}
\theoremstyle{theorem}\newtheorem{lemma}{Lemma}
\theoremstyle{definition}\newtheorem*{remark}{Remark}
\theoremstyle{definition}\newtheorem*{observation}{Observation}
\theoremstyle{definition}\newtheorem*{aside}{Aside}
\theoremstyle{definition}\newtheorem*{hint}{Hint}
\theoremstyle{theorem}\newtheorem*{cor}{Corollary}

\DeclareSymbolFont{bbold}{U}{bbold}{m}{n}
\DeclareSymbolFontAlphabet{\mathbbold}{bbold}

\newcommand{\into} {\hookrightarrow}
\newcommand {\onto} {\twoheadrightarrow}
\newcommand{\ip}[2]{\langle#1,#2\rangle}
\newcommand{\p}{\partial}
\newcommand{\pa}[2]{\frac{\p#1}{\p#2}}
\newcommand{\pat}[3]{\frac{\p^2#1}{\p#2\p#3}}
\newcommand{\closure}[1]{\overline{#1}}
\newcommand{\id}{\mathbbold{1}}
\newcommand{\im}{\operatorname{im}}
\newcommand{\inverse}{^{-1}}
\newcommand{\inv}{^{-1}}
\newcommand{\glnc}{\text{GL}_n(\mathbb{C})}
\newcommand{\slnc}{\text{SL}_n(\mathbb{C})}
\newcommand{\glnr}{\text{GL}_n(\mathbb{R})}
\newcommand{\slnr}{\text{SL}_n(\mathbb{R})}
\newcommand{\gl}{\text{GL}}
\newcommand{\msp}{\hspace*{.5em}}
\newcommand{\units}{^{\times}}
\newcommand{\sig}[3]{\sum\limits_{#1=#2}^{#3}}
\newcommand{\Span}{\text{span}}
\newcommand{\rep}{\text{Re }}
\newcommand{\imp}{\text{Im }}
\newcommand{\lcm}{\text{lcm}}
\newcommand{\bin}{\mathrm{Bin}}
\newcommand{\wkc}{\rightharpoonup}

\setlength{\parindent}{0pt} 
\setlength{\parskip}{2ex}
\DeclareGraphicsRule{.tif}{png}{.png}{`convert #1 `dirname #1`/`basename #1 .tif`.png}



%all of these can be empty



\student{Andrew Dong}
\classname{pset}
\instructor{}


\duedates{Blah}
\university{University of Chicago}
\term{} 
\psnumber{Blah}
%\psdescr{}  



\begin{document}
\maketitle 

\newpage
% Uncomment the \tableofcontents and \newpage lines to get a Contents page
% Uncomment the \setcounter line as well if you do NOT want subsections
%       listed in Contents
%\setcounter{tocdepth}{1}
%\tableofcontents
%\newpage

% When problems are long, it may be desirable to put a \newpage or a
% \clearpage before each homeworkProblem environment

Contents of this Study Guide:

A. {Basic Concepts}
\\ 1. Probability and Relative Frequency
\\ 2. Rudiments of Combinatorial Analysis

B. {Combination of Events}
\\ 3. Elementary Events.  The Sample Space
\\ 4. The Addition Law for Probabilities

C. {Dependent Events}
\\ 5. Conditional Probability
\\ 6. Statistical Independence

D. {Random Variables}
\\ 7. Discrete and Continuous Random Variables.  Distribution Functions
\\ 8. Mathematical Expectation
\\ 9. Chebyshev's Inequality.  The Variance and Correlation Coefficient

E. {Three Important Probability Distributions}
\\ 10. Bernoulli Trials.  The Binomial and Poisson Distributions
\\ 11.  The De Moivre-Laplace Theorem.  The Normal Distribution

F. {Some Limit Theorems}
\\ 12. The Law of Large Numbers
\\ 13.  Generating Functions.  Weak Convergence of Probability Distributions
\\ 14.  Characteristic Functions.  The Central Limit Theorem

G. {Markov Chains}
\\ 15. Transition Probabilities
\\ 16.  Persistent and Transient States
\\ 17.  Limiting Probabilities.  Stationary Distributions


H. {Continuous Markov Processes}
\\ 18.  Definitions.  The Sojourn Time
\\ 19.  The Kolmogorov Equations
\\ 20.  More on Limiting Probabilities.  Erlang's Formula

I. {Information Theory}

J. {Game Theory}

K. {Branching Processes}

L. {Problems of Optimal Control}

\clearpage

\section{ Basic Concepts }

\subsection {Probability and Relative Frequency}

\begin{equation}
P(A) = \frac{(N(A))}{N}
\end{equation}
where N is the total number of outcomes of the experiment and N(A) is the number of outcomes leading to the occurrence of event A



\end{document}






