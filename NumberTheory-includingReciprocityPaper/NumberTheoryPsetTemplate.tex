\documentclass{article}
% Change "article" to "report" to get rid of page number on title page
\usepackage{amsmath,amsfonts,amsthm,amssymb}
\usepackage[all]{xy}
\usepackage{enumerate,verbatim}
\usepackage{setspace}
\usepackage{Tabbing}
\usepackage{fancyhdr}
\usepackage{lastpage}
\usepackage{extramarks}
\usepackage{chngpage}
\usepackage{soul,color}
\usepackage{graphicx,float,wrapfig}

% In case you need to adjust margins:
\topmargin=-0.45in      %
\evensidemargin=0in     %
\oddsidemargin=0in      %
\textwidth=6.5in        %
\textheight=9.0in       %
\headsep=0.25in         %

% Homework Specific Information
\newcommand{\hmwkTitle}{Groups of Units and Quadratic Reciprocity}
\newcommand{\hmwkClass}{Number Theory}
\newcommand{\hmwkClassInstructor}{Professor Ngo}
\newcommand{\hmwkAuthorName}{Andrew Dong}

% Setup the header and footer
\pagestyle{fancy}                                                       %
\lhead{\hmwkAuthorName}                                                 %
\chead{\hmwkClass\ (\hmwkClassInstructor : \hmwkTitle)}  %
\rhead{\firstxmark}                                                     %
\lfoot{\lastxmark}                                                      %
\cfoot{}                                                                %
\rfoot{Page\ \thepage\ of\ \pageref{LastPage}}                          %
\renewcommand\headrulewidth{0.4pt}                                      %
\renewcommand\footrulewidth{0.4pt}                                      %

% This is used to trace down (pin point) problems
% in latexing a document:
%\tracingall

%%%%%%%%%%%%%%%%%%%%%%%%%%%%%%%%%%%%%%%%%%%%%%%%%%%%%%%%%%%%%
% Some tools
\newcommand{\enterProblemHeader}[1]{\nobreak\extramarks{#1}{#1 continued on next page\ldots}\nobreak%
                                    \nobreak\extramarks{#1 (continued)}{#1 continued on next page\ldots}\nobreak}%
\newcommand{\exitProblemHeader}[1]{\nobreak\extramarks{#1 (continued)}{#1 continued on next page\ldots}\nobreak%
                                   \nobreak\extramarks{#1}{}\nobreak}%

\newlength{\labelLength}
\newcommand{\labelAnswer}[2]
  {\settowidth{\labelLength}{#1}%
   \addtolength{\labelLength}{0.25in}%
   \changetext{}{-\labelLength}{}{}{}%
   \noindent\fbox{\begin{minipage}[c]{\columnwidth}#2\end{minipage}}%
   \marginpar{\fbox{#1}}%

   % We put the blank space above in order to make sure this
   % \marginpar gets correctly placed.
   \changetext{}{+\labelLength}{}{}{}}%

\setcounter{secnumdepth}{0}
\newcommand{\homeworkProblemName}{}%
\newcounter{homeworkProblemCounter}%
\newenvironment{homeworkProblem}[1][Problem \arabic{homeworkProblemCounter}]%
  {\stepcounter{homeworkProblemCounter}%
   \renewcommand{\homeworkProblemName}{#1}%
   \section{\homeworkProblemName}%
   \enterProblemHeader{\homeworkProblemName}}%
  {\exitProblemHeader{\homeworkProblemName}}%

\newcommand{\problemAnswer}[1]
  {\noindent\fbox{\begin{minipage}[c]{\columnwidth}#1\end{minipage}}}%

\newcommand{\problemLAnswer}[1]
  {\labelAnswer{\homeworkProblemName}{#1}}

\newcommand{\homeworkSectionName}{}%
\newlength{\homeworkSectionLabelLength}{}%
\newenvironment{homeworkSection}[1]%
  {% We put this space here to make sure we're not connected to the above.
   % Otherwise the changetext can do funny things to the other margin

   \renewcommand{\homeworkSectionName}{#1}%
   \settowidth{\homeworkSectionLabelLength}{\homeworkSectionName}%
   \addtolength{\homeworkSectionLabelLength}{0.25in}%
   \changetext{}{-\homeworkSectionLabelLength}{}{}{}%
   \subsection{\homeworkSectionName}%
   \enterProblemHeader{\homeworkProblemName\ [\homeworkSectionName]}}%
  {\enterProblemHeader{\homeworkProblemName}%

   % We put the blank space above in order to make sure this margin
   % change doesn't happen too soon (otherwise \sectionAnswer's can
   % get ugly about their \marginpar placement.
   \changetext{}{+\homeworkSectionLabelLength}{}{}{}}%

\newcommand{\sectionAnswer}[1]
  {% We put this space here to make sure we're disconnected from the previous
   % passage

   \noindent\fbox{\begin{minipage}[c]{\columnwidth}#1\end{minipage}}%
   \enterProblemHeader{\homeworkProblemName}\exitProblemHeader{\homeworkProblemName}%
   \marginpar{\fbox{\homeworkSectionName}}%

   % We put the blank space above in order to make sure this
   % \marginpar gets correctly placed.
   }%

%%%%%%%%%%%%%%%%%%%%%%%%%%%%%%%%%%%%%%%%%%%%%%%%%%%%%%%%%%%%%


%%%%%%%%%%%%%%%%%%%%%%%%%%%%%%%%%%%%%%%%%%%%%%%%%%%%%%%%%%%%%
% Make title
\title{\vspace{2in}\textmd{\textbf{\hmwkClass:\ \hmwkTitle}}\\\normalsize\vspace{0.1in}\vspace{0.1in}\large{\textit{\hmwkClassInstructor}}\vspace{3in}}
\date{}
\author{\textbf{\hmwkAuthorName}}
%%%%%%%%%%%%%%%%%%%%%%%%%%%%%%%%%%%%%%%%%%%%%%%%%%%%%%%%%%%%%
\newcommand{\inv}{^{-1}}
\begin{document}
\begin{spacing}{1.1}
\newpage
% Uncomment the \tableofcontents and \newpage lines to get a Contents page
% Uncomment the \setcounter line as well if you do NOT want subsections
%       listed in Contents
%\setcounter{tocdepth}{1}
%\tableofcontents
%\newpage

% When problems are long, it may be desirable to put a \newpage or a
% \clearpage before each homeworkProblem environment

\clearpage
\begin{homeworkProblem}[6.1]
\end{homeworkProblem}

\textbf{Question}: 
Find the orders of the elements of $U_9$ and of $U_10$.  

\textbf{Solution}:
\\Answer


\begin{homeworkProblem}[6.2]
\end{homeworkProblem}
\textbf{Question}:
Show that if l and m are positive integers with highest common factor h, then $gcd(2^l-1,2^m-1)$ divides $2^h-1$.  

\textbf{Solution}:
\\Answer

\begin{homeworkProblem}[6.3]
\end{homeworkProblem}
\textbf{Question}:
The groups $U_10$ and $U_12$ both have order 4; show that exactly one of them is cyclic.  

\textbf{Solution}:
\\Answer

\begin{homeworkProblem}[6.4]
\end{homeworkProblem}
\textbf{Question}:
Find primitive roots in $U_n$ for n = 18,23,27 and 31.  

\textbf{Solution}:
\\Answer

\begin{homeworkProblem}[6.5]
\end{homeworkProblem}
\textbf{Question}:
Show that if $U_n$ has a primitive root then it has $\phi(\phi(n))$ of them.  

\textbf{Solution}:
\\Answer

\begin{homeworkProblem}[6.6]
\end{homeworkProblem}
\textbf{Question}:
Verify that the element 5 is a generator of $U_7$

(answer to problem)

\begin{homeworkProblem}[6.7]
\end{homeworkProblem}
\textbf{Question}:
Find the elements of order d in $U_11$, for each d dividing 10; which elements are generators?  

\textbf{Solution}:
\\Answer

\begin{homeworkProblem}[6.8]
\end{homeworkProblem}
\textbf{Question}:
Verify that 2 is a primitive root mod(25) by calculating its powers.  

\textbf{Solution}:
\\Answer

\begin{homeworkProblem}[6.9]
\end{homeworkProblem}
\textbf{Question}:
Show that 2 is a primitive root mod $(3^e)$ for all $e\geqslant1$.  

\textbf{Solution}:
\\Answer

\begin{homeworkProblem}[6.10]
\end{homeworkProblem}
\textbf{Question}:
Find an integer which is a primitive root $mod(7^e)$ for all $e\geqslant1$.  

\textbf{Solution}:
\\Answer

\begin{homeworkProblem}[Problem 2]
\end{homeworkProblem}
\textbf{Question}:
Check that 3 is a primitive root modulo 17 by constructing an explicit isomorphism between $Z/16Z$ and $(Z/17Z)^x$ mapping the class of 1 on the class of 3.  Use this map to solve the congruence equations

\textbf{Solution}:
\\Answer

\begin{homeworkProblem}[(a)]
\end{homeworkProblem}

$z^{12} \equiv 16$ mod 17

\textbf{Solution}:
\\Answer

\begin{homeworkProblem}[(b)]
\end{homeworkProblem}

$x^{20} \equiv 13$ mod 17

\textbf{Solution}:
\\Answer

\begin{homeworkProblem}[(c)]
\end{homeworkProblem}

$x^{48} \equiv 9$ mod 17

\textbf{Solution}:
\\Answer

\begin{homeworkProblem}[(d)]
\end{homeworkProblem}

$x^{11} \equiv 9$ mod 17

\textbf{Solution}:
\\Answer

\begin{homeworkProblem}[7.1]
\end{homeworkProblem}
\textbf{Question}:
Find all solutions in $Z_{15}$ of the congruence $x^2 - 3x + 2 \equiv 0$ mod (15).  

\textbf{Solution}:
\\Answer

\begin{homeworkProblem}[7.2]
\end{homeworkProblem}
\textbf{Question}:
What square roots do the elements 5 and 16 have in $Z_{21}$?  Hence find all solutions of the congruences $x^2 + 3x + 1 \equiv 0$ mod (21) and $x^2 + 2x -3 \equiv 0$ mod (21).  

\textbf{Solution}:
\\Answer

\end{spacing}
\end{document}
