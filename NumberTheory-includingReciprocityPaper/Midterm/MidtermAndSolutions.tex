\documentclass{article}
% Change "article" to "report" to get rid of page number on title page
\usepackage{amsmath,amsfonts,amsthm,amssymb}
\usepackage[all]{xy}
\usepackage{enumerate,verbatim}
\usepackage{setspace}
\usepackage{Tabbing}
\usepackage{fancyhdr}
\usepackage{lastpage}
\usepackage{extramarks}
\usepackage{chngpage}
\usepackage{soul,color}
\usepackage{graphicx,float,wrapfig}

% In case you need to adjust margins:
\topmargin=-0.45in      %
\evensidemargin=0in     %
\oddsidemargin=0in      %
\textwidth=6.5in        %
\textheight=9.0in       %
\headsep=0.25in         %

% Homework Specific Information
\newcommand{\hmwkTitle}{Midterm Problems}
\newcommand{\hmwkClass}{Number Theory}
\newcommand{\hmwkClassInstructor}{Professor Ngo}
\newcommand{\hmwkAuthorName}{Andrew Dong}

% Setup the header and footer
\pagestyle{fancy}                                                       %
\lhead{\hmwkAuthorName}                                                 %
\chead{\hmwkClass\ (\hmwkClassInstructor : \hmwkTitle)}  %
\rhead{\firstxmark}                                                     %
\lfoot{\lastxmark}                                                      %
\cfoot{}                                                                %
\rfoot{Page\ \thepage\ of\ \pageref{LastPage}}                          %
\renewcommand\headrulewidth{0.4pt}                                      %
\renewcommand\footrulewidth{0.4pt}                                      %

% This is used to trace down (pin point) problems
% in latexing a document:
%\tracingall

%%%%%%%%%%%%%%%%%%%%%%%%%%%%%%%%%%%%%%%%%%%%%%%%%%%%%%%%%%%%%
% Some tools
\newcommand{\enterProblemHeader}[1]{\nobreak\extramarks{#1}{#1 continued on next page\ldots}\nobreak%
                                    \nobreak\extramarks{#1 (continued)}{#1 continued on next page\ldots}\nobreak}%
\newcommand{\exitProblemHeader}[1]{\nobreak\extramarks{#1 (continued)}{#1 continued on next page\ldots}\nobreak%
                                   \nobreak\extramarks{#1}{}\nobreak}%

\newlength{\labelLength}
\newcommand{\labelAnswer}[2]
  {\settowidth{\labelLength}{#1}%
   \addtolength{\labelLength}{0.25in}%
   \changetext{}{-\labelLength}{}{}{}%
   \noindent\fbox{\begin{minipage}[c]{\columnwidth}#2\end{minipage}}%
   \marginpar{\fbox{#1}}%

   % We put the blank space above in order to make sure this
   % \marginpar gets correctly placed.
   \changetext{}{+\labelLength}{}{}{}}%

\setcounter{secnumdepth}{0}
\newcommand{\homeworkProblemName}{}%
\newcounter{homeworkProblemCounter}%
\newenvironment{homeworkProblem}[1][Problem \arabic{homeworkProblemCounter}]%
  {\stepcounter{homeworkProblemCounter}%
   \renewcommand{\homeworkProblemName}{#1}%
   \section{\homeworkProblemName}%
   \enterProblemHeader{\homeworkProblemName}}%
  {\exitProblemHeader{\homeworkProblemName}}%

\newcommand{\problemAnswer}[1]
  {\noindent\fbox{\begin{minipage}[c]{\columnwidth}#1\end{minipage}}}%

\newcommand{\problemLAnswer}[1]
  {\labelAnswer{\homeworkProblemName}{#1}}

\newcommand{\homeworkSectionName}{}%
\newlength{\homeworkSectionLabelLength}{}%
\newenvironment{homeworkSection}[1]%
  {% We put this space here to make sure we're not connected to the above.
   % Otherwise the changetext can do funny things to the other margin

   \renewcommand{\homeworkSectionName}{#1}%
   \settowidth{\homeworkSectionLabelLength}{\homeworkSectionName}%
   \addtolength{\homeworkSectionLabelLength}{0.25in}%
   \changetext{}{-\homeworkSectionLabelLength}{}{}{}%
   \subsection{\homeworkSectionName}%
   \enterProblemHeader{\homeworkProblemName\ [\homeworkSectionName]}}%
  {\enterProblemHeader{\homeworkProblemName}%

   % We put the blank space above in order to make sure this margin
   % change doesn't happen too soon (otherwise \sectionAnswer's can
   % get ugly about their \marginpar placement.
   \changetext{}{+\homeworkSectionLabelLength}{}{}{}}%

\newcommand{\sectionAnswer}[1]
  {% We put this space here to make sure we're disconnected from the previous
   % passage

   \noindent\fbox{\begin{minipage}[c]{\columnwidth}#1\end{minipage}}%
   \enterProblemHeader{\homeworkProblemName}\exitProblemHeader{\homeworkProblemName}%
   \marginpar{\fbox{\homeworkSectionName}}%

   % We put the blank space above in order to make sure this
   % \marginpar gets correctly placed.
   }%

%%%%%%%%%%%%%%%%%%%%%%%%%%%%%%%%%%%%%%%%%%%%%%%%%%%%%%%%%%%%%


%%%%%%%%%%%%%%%%%%%%%%%%%%%%%%%%%%%%%%%%%%%%%%%%%%%%%%%%%%%%%
% Make title
\title{\vspace{2in}\textmd{\textbf{\hmwkClass:\ \hmwkTitle}}\\\normalsize\vspace{0.1in}\vspace{0.1in}\large{\textit{\hmwkClassInstructor}}\vspace{3in}}
\date{}
\author{\textbf{\hmwkAuthorName}}
%%%%%%%%%%%%%%%%%%%%%%%%%%%%%%%%%%%%%%%%%%%%%%%%%%%%%%%%%%%%%
\newcommand{\inv}{^{-1}}
\begin{document}
\begin{spacing}{1.1}
\newpage
% Uncomment the \tableofcontents and \newpage lines to get a Contents page
% Uncomment the \setcounter line as well if you do NOT want subsections
%       listed in Contents
%\setcounter{tocdepth}{1}
%\tableofcontents
%\newpage

% When problems are long, it may be desirable to put a \newpage or a
% \clearpage before each homeworkProblem environment
\begin{titlepage}
	\centering
	{\scshape\LARGE Elementary Number Theory \par}
	\vspace{1cm}
	{\scshape\Large Midterm Review\par}
	\vspace{1.5cm}
	{\huge\bfseries Number Theory Midterm and Solutions\par}
	\vspace{1cm}
	{\Large\itshape Andrew Dong\par}
	\vspace{2cm}

	\vfill


% Bottom of the page
	Professor~Ngo \textsc{Bau Chau}
	\\ Minh-Tam Trinh
	\vspace{5 mm}
	\\{\large \today\par}
\end{titlepage}


\clearpage

\section{Math 17500: Midterm}

\subsection{1.  Find all positive integers x less than 200 such that $x \equiv 1$ mod 11 and $x \equiv 9$ mod 13}

Notice that gcd(11,13) = 1 which implies that the solution belongs to the congruence class

$$ x \equiv 1 * 13 * a + 9 * 11 * b \text{ mod 11} * 13$$

where a,b $\in$ Z satisfying 13a $\equiv$ 1 mod 11 and 
\\11b $\equiv$ 1 mod 13.  
\\The equation $13a \equiv 2a \equiv 1$ mod 11 has solution $a \equiv$ 6 mod 11.  
\\The equation $11b \equiv -2b \equiv 1$ mod 13 has solution b $\equiv -7 \equiv 6$ mod 13.  
\\Therefore, $x \equiv 13 * 6 + 9 * 11 * 6 \equiv 100$ mod 143.  
\\In the range $0 < x < 200$, the only solution is x = 100.  



\vfill

\subsection{2.  Find all positive Integers less than 100 such that $x^2 \equiv 11$ mod 49}

We first find the solution of the equation $x^2 - 11 \equiv x^2 - 40$ mod 7.  This equation has two solutions: $x \equiv \pm 2$ mod 7.  
\\The solution of the equation $x^2 - 11 \equiv 0$ mod 49 then must have the form $x \equiv 2$ + 7y.  
\\If x = 2 + 7y, we have $(2 + 7y)^2 - 11 \equiv 28y - 7$ mod 49.  This is equivalent to
\\$4y \equiv 1$ mod 7 and 
\\$y \equiv 2$ mod 7.  
\\In this case $x \equiv 16$ mod 49.  
\\If x = -2 + 7y, a similar calculation implies 

\vfill

\subsection{3.  Find the residue of $2^{1000} + 2^{100}$ modulo 13}

\vfill

\subsection{4.  Check that 2 is a primitive root modulo 13 by calculating the residue modulo 13 of all powers of 2}

\vfill

\subsection{5.  Find all residue classes x modulo 13 such that $x^3 \equiv 1$ mod 13}

\vfill

\subsection{6.  Find all residue classes x modulo 169 such that $x^3 \equiv 1$ mod 169}

\vfill

\subsection{7.  Prove that $(Z/5Z)^x and (Z/8z)^x$ are not isomorphic as abelian groups}

\vfill

\subsection{8.  Prove that $2^n + 1$ is a prime if and only if $\phi(2^n + 1) = 2^n$}

\vfill

\subsection{9.  Prove that $(Z/(2^{n + 1})Z)^x$ and $(Z/2^{2^n+1}Z)^x$ are not isomorphic as abelian groups} 

\vfill

\subsection{10.  Let p be an odd prime.  How many primitive roots modulo $p^2$ are there? Justify your answer} 

\vfill

\end{spacing}
\end{document}
