\documentclass{article}
% Change "article" to "report" to get rid of page number on title page
\usepackage{amsmath,amsfonts,amsthm,amssymb}
\usepackage[all]{xy}
\usepackage{enumerate,verbatim}
\usepackage{setspace}
\usepackage{Tabbing}
\usepackage{fancyhdr}
\usepackage{lastpage}
\usepackage{extramarks}
\usepackage{chngpage}
\usepackage{soul,color}
\usepackage{graphicx,float,wrapfig}

% In case you need to adjust margins:
\topmargin=-0.45in      %
\evensidemargin=0in     %
\oddsidemargin=0in      %
\textwidth=6.5in        %
\textheight=9.0in       %
\headsep=0.25in         %

% Homework Specific Information
\newcommand{\hmwkTitle}{Groups of Units and Quadratic Reciprocity}
\newcommand{\hmwkClass}{Number Theory}
\newcommand{\hmwkClassInstructor}{Professor Ngo}
\newcommand{\hmwkAuthorName}{Andrew Dong}

% Setup the header and footer
\pagestyle{fancy}                                                       %
\lhead{\hmwkAuthorName}                                                 %
\chead{\hmwkClass\ (\hmwkClassInstructor : \hmwkTitle)}  %
\rhead{\firstxmark}                                                     %
\lfoot{\lastxmark}                                                      %
\cfoot{}                                                                %
\rfoot{Page\ \thepage\ of\ \pageref{LastPage}}                          %
\renewcommand\headrulewidth{0.4pt}                                      %
\renewcommand\footrulewidth{0.4pt}                                      %

% This is used to trace down (pin point) problems
% in latexing a document:
%\tracingall

%%%%%%%%%%%%%%%%%%%%%%%%%%%%%%%%%%%%%%%%%%%%%%%%%%%%%%%%%%%%%
% Some tools
\newcommand{\enterProblemHeader}[1]{\nobreak\extramarks{#1}{#1 continued on next page\ldots}\nobreak%
                                    \nobreak\extramarks{#1 (continued)}{#1 continued on next page\ldots}\nobreak}%
\newcommand{\exitProblemHeader}[1]{\nobreak\extramarks{#1 (continued)}{#1 continued on next page\ldots}\nobreak%
                                   \nobreak\extramarks{#1}{}\nobreak}%

\newlength{\labelLength}
\newcommand{\labelAnswer}[2]
  {\settowidth{\labelLength}{#1}%
   \addtolength{\labelLength}{0.25in}%
   \changetext{}{-\labelLength}{}{}{}%
   \noindent\fbox{\begin{minipage}[c]{\columnwidth}#2\end{minipage}}%
   \marginpar{\fbox{#1}}%

   % We put the blank space above in order to make sure this
   % \marginpar gets correctly placed.
   \changetext{}{+\labelLength}{}{}{}}%

\setcounter{secnumdepth}{0}
\newcommand{\homeworkProblemName}{}%
\newcounter{homeworkProblemCounter}%
\newenvironment{homeworkProblem}[1][Problem \arabic{homeworkProblemCounter}]%
  {\stepcounter{homeworkProblemCounter}%
   \renewcommand{\homeworkProblemName}{#1}%
   \section{\homeworkProblemName}%
   \enterProblemHeader{\homeworkProblemName}}%
  {\exitProblemHeader{\homeworkProblemName}}%

\newcommand{\problemAnswer}[1]
  {\noindent\fbox{\begin{minipage}[c]{\columnwidth}#1\end{minipage}}}%

\newcommand{\problemLAnswer}[1]
  {\labelAnswer{\homeworkProblemName}{#1}}

\newcommand{\homeworkSectionName}{}%
\newlength{\homeworkSectionLabelLength}{}%
\newenvironment{homeworkSection}[1]%
  {% We put this space here to make sure we're not connected to the above.
   % Otherwise the changetext can do funny things to the other margin

   \renewcommand{\homeworkSectionName}{#1}%
   \settowidth{\homeworkSectionLabelLength}{\homeworkSectionName}%
   \addtolength{\homeworkSectionLabelLength}{0.25in}%
   \changetext{}{-\homeworkSectionLabelLength}{}{}{}%
   \subsection{\homeworkSectionName}%
   \enterProblemHeader{\homeworkProblemName\ [\homeworkSectionName]}}%
  {\enterProblemHeader{\homeworkProblemName}%

   % We put the blank space above in order to make sure this margin
   % change doesn't happen too soon (otherwise \sectionAnswer's can
   % get ugly about their \marginpar placement.
   \changetext{}{+\homeworkSectionLabelLength}{}{}{}}%

\newcommand{\sectionAnswer}[1]
  {% We put this space here to make sure we're disconnected from the previous
   % passage

   \noindent\fbox{\begin{minipage}[c]{\columnwidth}#1\end{minipage}}%
   \enterProblemHeader{\homeworkProblemName}\exitProblemHeader{\homeworkProblemName}%
   \marginpar{\fbox{\homeworkSectionName}}%

   % We put the blank space above in order to make sure this
   % \marginpar gets correctly placed.
   }%

%%%%%%%%%%%%%%%%%%%%%%%%%%%%%%%%%%%%%%%%%%%%%%%%%%%%%%%%%%%%%


%%%%%%%%%%%%%%%%%%%%%%%%%%%%%%%%%%%%%%%%%%%%%%%%%%%%%%%%%%%%%
% Make title
\title{\vspace{2in}\textmd{\textbf{\hmwkClass:\ \hmwkTitle}}\\\normalsize\vspace{0.1in}\vspace{0.1in}\large{\textit{\hmwkClassInstructor}}\vspace{3in}}
\date{}
\author{\textbf{\hmwkAuthorName}}
%%%%%%%%%%%%%%%%%%%%%%%%%%%%%%%%%%%%%%%%%%%%%%%%%%%%%%%%%%%%%
\newcommand{\inv}{^{-1}}
\begin{document}
\begin{spacing}{1.1}
\newpage
% Uncomment the \tableofcontents and \newpage lines to get a Contents page
% Uncomment the \setcounter line as well if you do NOT want subsections
%       listed in Contents
%\setcounter{tocdepth}{1}
%\tableofcontents
%\newpage

% When problems are long, it may be desirable to put a \newpage or a
% \clearpage before each homeworkProblem environment

\clearpage
\begin{homeworkProblem}[6.1]
\end{homeworkProblem}

\textbf{Question}: 
Find the orders of the elements of $U_9$ and of $U_{10}$.  

\textbf{Solution}:
\\$U_9 = \{1,2,4,5,7,8\}$ which have order respectively 1,6,3,6,3,2
\\$U_{10} = \{1,3,7,9\}$ which have order respectively 1,4,4,2



\begin{homeworkProblem}[6.2]
\end{homeworkProblem}
\textbf{Question}:
Show that if l and m are positive integers with highest common factor h, then $gcd(2^l-1,2^m-1)$ divides $2^h-1$.  

\textbf{Solution}:
\\Let k be the order of the element 2 in the group $U_n$.  Since h divides $2^l-1$, $2^l=1$ in $U_n$ which implies that $k|l$.  Similarly k divides m, so $k|gcd(l,m)=h$.  Then $2^n=1$ in $U_n$ since $2^k=1$ and $k|h$ which implies that $n|{2^n-1}$.  $\qed$




\begin{homeworkProblem}[6.3]
\end{homeworkProblem}
\textbf{Question}:
The groups $U_{10}$ and $U_{12}$ both have order 4; show that exactly one of them is cyclic.  

\textbf{Solution}:
\\By Homework Problem 6.1 we know that the elements $\{1,3,7,9\}$ of $U_{10}$ are generated by 3 since $3^4=1$, $3^1=3$, $3^3=7$, $3^2=9$.  Thus $U_{10}$ is generated by 3.  In $U_{12}$ $1^2,5^2,7^2,11^2=1$, thus no element has order $\phi(12)=4$.  

\begin{homeworkProblem}[6.4]
\end{homeworkProblem}
\textbf{Question}:
Find primitive roots in $U_n$ for n = 18, 23, 27 and 31.  

\textbf{Solution}:
\\ Recall that by a previous Lemma, we have that an element $a \in U_n$ is a primitive root if and only if $a^{\frac{\phi(n)}{q}}\neq1$ in $U_n$ for each q dividing $\phi(n)$.  

For the case of n = 18 we consider a = 5 since a=2, a=3, a=4 are not units mod(18).  Meanwhile, $5^{\frac{\phi(18)}{q}}\neq1$ in $U_18$ for q dividing $\phi(18)$

For n = 23 we take a = 5 again since a=2, a=3, a=4 are not units mod(23).  $5^{\frac{\phi(23)}{q}}\neq1$ in $U_23$ for q dividing $\phi(23)$

For n = 27 we can instead take a = 2 since a=2 is a unit mod(27).  Furthermore $2^{\frac{\phi(27)}{q}}\neq1$ in $U_27$ for q dividing $\phi(27)$

For n = 31 we can not take a = 2 but instead must go to a = 3 to get $3^{\frac{\phi(31)}{q}}\neq1$ in $U_31$ for q dividing $\phi(31)$


\begin{homeworkProblem}[6.5]
\end{homeworkProblem}
\textbf{Question}:
Show that if $U_n$ has a primitive root then it has $\phi(\phi(n))$ of them.  

\textbf{Solution}:

Suppose a is a primitive root of $U_n$.  Then we know that $U_n$ is cyclic and can be generated by a.  The order of $U_n$ can be written as $m=\phi(n)$.  Thus, we see that $U_n$ can be generated by $a^k$ if and only if k and m are relatively prime.  The number of primitive roots $a^k$ is $\phi(m)=\phi(\phi(n))$.   $\qed$

\begin{homeworkProblem}[6.6]
\end{homeworkProblem}
\textbf{Question}:
Verify that the element 5 is a generator of $U_7$

To verify that the element 5 is a generator of $U_7$, consider the powers of 5 in $U_7$ : $5^1 = 5, 5^2 = 4, 5^3 = 6, 5^4 = 2, 5^5 = 3, 5^6 = 1$.  Thus, every element of $U_7$ can be written as a power of 5, which implies that the element 5 generates $U_7$.  $\qed$

\begin{homeworkProblem}[6.7]
\end{homeworkProblem}
\textbf{Question}:
Find the elements of order d in $U_{11}$, for each d dividing 10; which elements are generators?  

\textbf{Solution}:

Elements which divide 10 are: $\{1,2,5,10\}$ and the elements of order d form the sets $\{1\}$, $\{10\}$, $\{3,4,5,9\}$, and $\{2,6,7,8\}$.  The generators are $\{2,6,7,8\}$.  

\begin{homeworkProblem}[6.8]
\end{homeworkProblem}
\textbf{Question}:
Verify that 2 is a primitive root mod(25) by calculating its powers.  

\textbf{Solution}:

To verify that 2 is a primitive root mod(25), we consider the powers of 2 in the unit group $U_25$.  The powers of 2 are $\{2,4,8,16,7,14,3,6,12,24 = -1,-2 = 23, -4 = 21, -8 = 17, -16 = 9, 18, 11, 22 = -3, -6 = 19, -12 = 13, 1\}$.  Thus 2 has order 20 = $\phi(25)$ which implies that 2 is a primitive root mod(25).  

\begin{homeworkProblem}[6.9]
\end{homeworkProblem}
\textbf{Question}:
Show that 2 is a primitive root mod $(3^e)$ for all $e\geqslant1$.  

\textbf{Solution}:

To show that 2 is a primitive root mod $(3^e)$ given $e\geqslant1$ we first consider it as a primitive root mod $(3^2)$.  If we can show that 2 is a primitive root mod $(3^2)$  in $U_{3^2}$, it will follow that it is also a primitive root mod ($3^e$) for all e.  Now, 2 has order $\phi(3^2) = 6$ in $U_{3^2}$ which implies that 2 is a primitive root mod($3^2$).  Thus, we conclude that 2 is a primitive root mod $(3^e)$ for all $e\geqslant1$.  $\qed$

\begin{homeworkProblem}[6.10]
\end{homeworkProblem}
\textbf{Question}:
Find an integer which is a primitive root $mod(7^e)$ for all $e\geqslant1$.  

\textbf{Solution}:

3 is a primitive root mod(7).  $3^6 = 729 \neq 1$ mod$(7^2)$ thus 3 is a primitive root mod $(7^e)$ for e = 2 and therefore all e.  

\begin{homeworkProblem}[Problem 2]
\end{homeworkProblem}
\textbf{Question}:
Check that 3 is a primitive root modulo 17 by constructing an explicit isomorphism between $Z/16Z$ and $(Z/17Z)^x$ mapping the class of 1 on the class of 3.  Use this map to solve the congruence equations

\textbf{Solution}:

$3^1 = 3 \neq 1, 3^2 = 9 \neq 1, 3^4 = (3^2)^2 = 81 = 13 \neq 1, 3^8 = (3^4)^2 = 13^2 = 169 = 16 \neq 1.$.  By Fermat's little theorem and Lagrange Theorem, 3 is a primitive root modulo 17.  

\begin{homeworkProblem}[(a)]
\end{homeworkProblem}

$z^{12} \equiv 16$ mod 17

\textbf{Solution}:

First note that any solution z must be a unit mod (17), so z, like 16 is an element of $U_17$.  By corollary, this group is cyclic so both z and 16 can be expressed as powers of a primitive root g mod(17).  Since we know that 3 is a primitive root mod(17) we take g = 3.  The powers of 3 (mod 17) are 3,9,10,13,15,11,16,15,8,7,4,12,2,6,1.  We see that $3^7 = 16$ in $U_17$ so we write z = $3^i$ where the exponent i is unknown.  Then $z^{12} = 3^{12i}$ so our congruence becomes $3^{12i}$ = $3^{12}$ in $U_17$.  3, because it is a primitive root has order $\phi(17) = 16$ so $3^{12i} = 3^{12}$ if and only if $12i \equiv 12$ mod(16) or equivalently $i \equiv 1$ mod (16).  The relevant values of i are 1 and 13 so the solutions of the original congruence are $z \equiv 3, 3^13$ mod (17).  $3^13 \equiv 2$.  There are two congruence classes of solutions, namely $z \equiv 3, 2$ mod(17).  

\begin{homeworkProblem}[(b)]
\end{homeworkProblem}

$x^{20} \equiv 13$ mod 17

\textbf{Solution}:

Any solution x must be a unit mod (17), so x, like 13 is an element of $U_17$.  This group is cyclic so both x and 13 can be expressed as powers of a primitive root 3 mod(17).  The powers of 3 (mod 17) are again:  3,9,10,13,15,11,16,15,8,7,4,12,2,6,1.  We see that $3^4 = 13$ in $U_17$ so we write z = $3^i$ where the exponent i is unknown.  Then $x^{20} = 3^{20i}$ so our congruence becomes $3^{20i}$ = $3^{20}$ in $U_17$.  $3^{20i} = 3^{20}$ if and only if $20i \equiv 20$ mod(16) or equivalently $i \equiv 1$ mod (16).  The relevant values of i are 1, 5, 9 and 13 so the solutions of the original congruence are $x \equiv 3, 3^4$ mod (17).  $3^4 \equiv 13$.  We further notice that $x \equiv -3 \equiv 14 and x \equiv -5 \equiv 12$.  Thus there are four congruence classes of solutions, $x \equiv 3,5,12 and 14$ mod (17).   

\begin{homeworkProblem}[(c)]
\end{homeworkProblem}

$x^{48} \equiv 9$ mod 17

\textbf{Solution}:

Any solution x must be a unit mod (17), so x, like 9 is an element of $U_17$.  This group is cyclic so both x and 9 can be expressed as powers of a primitive root 3 mod(17).  The powers of 3 (mod 17) are  3,9,10,13,15,11,16,15,8,7,4,12,2,6,1.  We see that $3^2 = 9$ in $U_17$ so we write z = $3^i$ where the exponent i is unknown.  Then $x^{48} = 3^{48i}$ so our congruence becomes $3^{48i}$ = $3^{48}$ in $U_17$.  $3^{48i} = 3^{48}$ if and only if $48i \equiv 48$ mod(16) or equivalently $i \equiv 1$ mod (16).  The relevant values of i are 1, 3, 5, 7, 9, 11, 13 and 15 so the solutions of the original congruence are $x \equiv 3, 3^2$ mod (17).  $3^2 \equiv 9$.  There are 8 congruence classes of solutions.  

\begin{homeworkProblem}[(d)]
\end{homeworkProblem}

$x^{11} \equiv 9$ mod 17

\textbf{Solution}:

Solutions of x must be unit mod (17), so x and 9 are elements of $U_17$.  This group is cyclic so both x and 9 can be expressed as powers of a primitive root 3 mod(17).  The powers of 3 (mod 17) are  3,9,10,13,15,11,16,15,8,7,4,12,2,6,1.  We see that $3^2 = 9$ in $U_17$ so we write z = $3^i$ where the exponent i is unknown.  Then $x^{11} = 3^{11i}$ so our congruence becomes $3^{11i}$ = $3^{11}$ in $U_17$.  $3^{11i} = 3^{11}$ if and only if $11i \equiv 11$ mod(16) or equivalently $i \equiv 1$ mod (16).  The relevant values of i are 1, 3, 5, 7, 9, 11, 13 and 15 so the solutions of the original congruence are $x \equiv 3, 3^2$ mod (17).  $3^2 \equiv 9$.  There are 8 congruence classes of solutions.  


\begin{homeworkProblem}[7.1]
\end{homeworkProblem}
\textbf{Question}:
Find all solutions in $Z_{15}$ of the congruence $x^2 - 3x + 2 \equiv 0$ mod (15).  

\textbf{Solution}:

Recall the formula
\begin{equation}
x = \frac{-b \pm \sqrt[]{b^2-4ac}}{2a}
\end{equation}
which can be rewritten 
\begin{equation}
(2ax+b)^2 = b^2-4ac
\end{equation}
Applying formula we get that x = 1,2,7,11 in $Z_{15}$

\begin{homeworkProblem}[7.2]
\end{homeworkProblem}
\textbf{Question}:
What square roots do the elements 5 and 16 have in $Z_{21}$?  Hence find all solutions of the congruences $x^2 + 3x + 1 \equiv 0$ mod (21) and $x^2 + 2x -3 \equiv 0$ mod (21).  

\textbf{Solution}:
5 has no square roots in $Z_{21}$ and therefore no solutions.  16 has square roots $\pm4, \pm10$ and therefore has solutions 1,-3,4 and -6.  

\end{spacing}
\end{document}
