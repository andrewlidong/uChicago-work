\documentclass{article}
% Change "article" to "report" to get rid of page number on title page
\usepackage{amsmath,amsfonts,amsthm,amssymb}
\usepackage[all]{xy}
\usepackage{enumerate,verbatim}
\usepackage{setspace}
\usepackage{Tabbing}
\usepackage{fancyhdr}
\usepackage{lastpage}
\usepackage{extramarks}
\usepackage{chngpage}
\usepackage{soul,color}
\usepackage{graphicx,float,wrapfig}

% In case you need to adjust margins:
\topmargin=-0.45in      %
\evensidemargin=0in     %
\oddsidemargin=0in      %
\textwidth=6.5in        %
\textheight=9.0in       %
\headsep=0.25in         %

% Homework Specific Information
\newcommand{\hmwkTitle}{Simultaneous Linear Congruences}
\newcommand{\hmwkClass}{Number Theory}
\newcommand{\hmwkClassInstructor}{Professor Ngo}
\newcommand{\hmwkAuthorName}{Andrew Dong}

% Setup the header and footer
\pagestyle{fancy}                                                       %
\lhead{\hmwkAuthorName}                                                 %
\chead{\hmwkClass\ (\hmwkClassInstructor : \hmwkTitle)}  %
\rhead{\firstxmark}                                                     %
\lfoot{\lastxmark}                                                      %
\cfoot{}                                                                %
\rfoot{Page\ \thepage\ of\ \pageref{LastPage}}                          %
\renewcommand\headrulewidth{0.4pt}                                      %
\renewcommand\footrulewidth{0.4pt}                                      %

% This is used to trace down (pin point) problems
% in latexing a document:
%\tracingall

%%%%%%%%%%%%%%%%%%%%%%%%%%%%%%%%%%%%%%%%%%%%%%%%%%%%%%%%%%%%%
% Some tools
\newcommand{\enterProblemHeader}[1]{\nobreak\extramarks{#1}{#1 continued on next page\ldots}\nobreak%
                                    \nobreak\extramarks{#1 (continued)}{#1 continued on next page\ldots}\nobreak}%
\newcommand{\exitProblemHeader}[1]{\nobreak\extramarks{#1 (continued)}{#1 continued on next page\ldots}\nobreak%
                                   \nobreak\extramarks{#1}{}\nobreak}%

\newlength{\labelLength}
\newcommand{\labelAnswer}[2]
  {\settowidth{\labelLength}{#1}%
   \addtolength{\labelLength}{0.25in}%
   \changetext{}{-\labelLength}{}{}{}%
   \noindent\fbox{\begin{minipage}[c]{\columnwidth}#2\end{minipage}}%
   \marginpar{\fbox{#1}}%

   % We put the blank space above in order to make sure this
   % \marginpar gets correctly placed.
   \changetext{}{+\labelLength}{}{}{}}%

\setcounter{secnumdepth}{0}
\newcommand{\homeworkProblemName}{}%
\newcounter{homeworkProblemCounter}%
\newenvironment{homeworkProblem}[1][Problem \arabic{homeworkProblemCounter}]%
  {\stepcounter{homeworkProblemCounter}%
   \renewcommand{\homeworkProblemName}{#1}%
   \section{\homeworkProblemName}%
   \enterProblemHeader{\homeworkProblemName}}%
  {\exitProblemHeader{\homeworkProblemName}}%

\newcommand{\problemAnswer}[1]
  {\noindent\fbox{\begin{minipage}[c]{\columnwidth}#1\end{minipage}}}%

\newcommand{\problemLAnswer}[1]
  {\labelAnswer{\homeworkProblemName}{#1}}

\newcommand{\homeworkSectionName}{}%
\newlength{\homeworkSectionLabelLength}{}%
\newenvironment{homeworkSection}[1]%
  {% We put this space here to make sure we're not connected to the above.
   % Otherwise the changetext can do funny things to the other margin

   \renewcommand{\homeworkSectionName}{#1}%
   \settowidth{\homeworkSectionLabelLength}{\homeworkSectionName}%
   \addtolength{\homeworkSectionLabelLength}{0.25in}%
   \changetext{}{-\homeworkSectionLabelLength}{}{}{}%
   \subsection{\homeworkSectionName}%
   \enterProblemHeader{\homeworkProblemName\ [\homeworkSectionName]}}%
  {\enterProblemHeader{\homeworkProblemName}%

   % We put the blank space above in order to make sure this margin
   % change doesn't happen too soon (otherwise \sectionAnswer's can
   % get ugly about their \marginpar placement.
   \changetext{}{+\homeworkSectionLabelLength}{}{}{}}%

\newcommand{\sectionAnswer}[1]
  {% We put this space here to make sure we're disconnected from the previous
   % passage

   \noindent\fbox{\begin{minipage}[c]{\columnwidth}#1\end{minipage}}%
   \enterProblemHeader{\homeworkProblemName}\exitProblemHeader{\homeworkProblemName}%
   \marginpar{\fbox{\homeworkSectionName}}%

   % We put the blank space above in order to make sure this
   % \marginpar gets correctly placed.
   }%

%%%%%%%%%%%%%%%%%%%%%%%%%%%%%%%%%%%%%%%%%%%%%%%%%%%%%%%%%%%%%


%%%%%%%%%%%%%%%%%%%%%%%%%%%%%%%%%%%%%%%%%%%%%%%%%%%%%%%%%%%%%
% Make title
\title{\vspace{2in}\textmd{\textbf{\hmwkClass:\ \hmwkTitle}}\\\normalsize\vspace{0.1in}\vspace{0.1in}\large{\textit{\hmwkClassInstructor}}\vspace{3in}}
\date{}
\author{\textbf{\hmwkAuthorName}}
%%%%%%%%%%%%%%%%%%%%%%%%%%%%%%%%%%%%%%%%%%%%%%%%%%%%%%%%%%%%%
\newcommand{\inv}{^{-1}}
\begin{document}
\begin{spacing}{1.1}
\newpage
% Uncomment the \tableofcontents and \newpage lines to get a Contents page
% Uncomment the \setcounter line as well if you do NOT want subsections
%       listed in Contents
%\setcounter{tocdepth}{1}
%\tableofcontents
%\newpage

% When problems are long, it may be desirable to put a \newpage or a
% \clearpage before each homeworkProblem environment
\begin{titlepage}
	\centering
	{\scshape\LARGE Elementary Number Theory \par}
	\vspace{1cm}
	{\scshape\Large Problem Set: Week 6\par}
	\vspace{1.5cm}
	{\huge\bfseries A selection of exercises on Simultaneous Linear Congruences and supplementary exercises on Quadratic Residues\par}
	\vspace{1cm}
	{\Large\itshape Andrew Dong\par}
	\vspace{2cm}

	\vfill


% Bottom of the page
	Professor~Ngo \textsc{Bao Chau}
	\\ Minh-Tam Trinh
	\vspace{5 mm}
	\\{\large \today\par}
\end{titlepage}


\clearpage

\section{Math 17500: Problem Set Week 6}

\subsection{1.  Find all integers satisfying both congruences $x \equiv 10$ mod 24 and $x \equiv 16$ mod 18}

\textbf{Solution: }

$x \equiv 10$ mod 24 $\Rightarrow x = 10$ + 24t.  
\\ Putting this value of x into our second congruence $x \equiv 16$ mod 18 we get
\\$10 + 24t \equiv 16$ mod 18 which becomes $24t \equiv 6$ mod 18 which becomes $4t \equiv 1$ mod 3 which can be rewritten $t \equiv 1$ mod 3.  Thus t = 1 + 3s where $s \in Z$.  

Plugging in this value for t into our original congruence, we get $x = 10 + 24(1+3s) = 34 + 72s$.  Thus $x \equiv 34$ mod 72.  Notice that 72 is the lcm(24,28).  

\vfill

\subsection{2.  Find all integers satisfying both congruences $x\ \equiv 8$ mod 9 and $x \equiv 31$ mod 33}

\textbf{Solution: }

$x \equiv 8$ mod 9 implies that x = 8 + 9t for integer valued t.  Plugging this value of x into our second congruence yields the congruence $8 + 9t \equiv 31$ mod 33 or $9t \equiv 23$ mod 33.  Since gcd(9,33) = 3 and 3 does not divide 23, t has no solutions, thus there are no integers satisfying both congruences. 


\vfill

\subsection{3.  Prove that there exists integer $x \in Z$ satisfying $x \equiv m$ mod a and $x \equiv n$ mod b if and only if $m \equiv n$ mod gcd(a,b).  In that case, find the general form of the solution}

\textbf{Solution: }

If an integer solution x exists then $x \equiv m$ mod a and $x \equiv n$ mod b and thus $a | (x-m)$ and $b | (x - n)$.  Let c = gcd(a,b) so c divides both a and b and therefore also divides x-m and x-n.  Notice that c must divide (x-n) - (x-m) = m - n, which is equivalent to saying that $m = n$ mod c $\sim$ $ m\equiv n$ mod gcd(a,b).  

The general solution forms a single congruence class mod(y) where y = lcm(a,b).  Suppose $x_0$ is any solution of the congruences.  Then an integer x is a solution to the congruences if and only if $x \equiv x_0$ mod (a) and $x \equiv x_0$ mod (b).  This implies that $x - x_0$ is divisible by a and b, or equivalently $x - x_0$ is divisible by their least common multiple lcm(a,b).  Thus, the general solution consists of a single congruence clas $x_0$ mod lcm(a,b).  

\vfill

\subsection{4.  Solve the system of congruences: 
\\$ 2x + 36 \equiv 1$ mod 17
\\$ 5x + 10y \equiv 2$ mod 17}

\textbf{Solution: }

$2x \equiv 16$ mod 17 $\implies$ $x \equiv 8$ mod 16.  

We then have $40 + 10y \equiv 2$ mod 17 $\implies$ $10y \equiv 13$ mod 17 $\implies$ [y] = [3] since $3 * 10 = 30 = 13$ mod 17.  


\vfill

\subsection{5.  Solve the system of congruences: 
\\$ 2x + 3y \equiv 1$ mod 24
\\$ 6x + 10y \equiv 2$ mod 24}

\textbf{Solution: }

we write our first congruence
\\$6x + 9y \equiv 3$ mod 24.  Subtracting our second congruence from this transformation of our first congruence we get $y \equiv 23$ mod 24.  
\\Putting this value back into our first congruence we get $2x - 3 \equiv 1$ mod 24.  Thus $2x \equiv 4$ mod 24 and $x \equiv 2$ mod 12.  

\vfill

\subsection{6.  Solve the congruence equation $x^2 \equiv 61$ mod 100}

\textbf{Solution: }

The congruence equation has solutions for $x \equiv 19$ mod 50 and $ \equiv 31$ mod 50.  

\vfill

\subsection{7.  Solve the congruence equation $x^2 \equiv 61$ mod 1000}

\textbf{Solution: }

Actually, this congruence equation doesn't have any solutions because $x \equiv 19$ mod 50 and $ \equiv 31$ mod 50 both fail mod for  $x^2 \equiv 61$ mod 1000.  

\vfill

\subsection{8.  Exercise 7.20}

\textbf{Question: }Show that, for each $r \geq 1$, there are infinitely many primes $p \equiv 1$ mod ($2^r)$.  

\textbf{Solution: }

Suppose there are instead finitely many primes $p \equiv 1$ mod ($2^r$).  Name them $p_1,...,p_k$ and define a = $2p_1*...*p_k$ and $m = a^{2^{2-1}} + 1$ which is divisible by an odd prime p.  Since a has order $2^r$ in $U_p$, by Lagrange's theorem we have that $2^r | p-1$.  This implies the congruence $p \equiv 1 mod (2^r)$ which means that p = $p_i$ for some i and p divides a.  But since p divies m, we have that $p|m - a^{2{r-1}} = 1$, which is a contradiction.  Thus, there must be infinitely many primes $p \equiv 1$ mod ($2^r$).  $\qed$


\vfill

\subsection{9.  Exercise 7.21} 

\textbf{Question: }For which values of n is -1 a quadratic residue mod (n)?

\textbf{Solution: }

We determine the values for which -1 is a quadratic residue mod n by our corollary that $-1 \in Q_p \iff p \equiv 1$ mod 4.  However, we know that a value n is in the set of quadratic residues over n if and only if a $\in$ the set of quadratic residues over $n_i$ for where $n = n_1*n_2*...*n_i$ where $n_i$ are mutually coprime.  

To determine values n can take, we must check the cases for two quadratic residue sets, $Q_{2^e}$ and $Q_{n_i}$ where $n_i > 2$.  For the case of $Q_{2^e}$ we know that e must equal 0 or 1 for $-1 \in Q_{2^e}$.  For the case of $Q_{n_i}$ where $n_i > 2$ we apply our corollary and claim that $-1 \in Q_{n_i}$ where $n_i > 2 \iff n_1 \equiv 1 mod 4$.  

Taking the results from these two cases we conclude that -1 is a quadratic residue mod(n) for values n not divisible by 4 or any prime of the form $p\equiv 3$ mod 4.  

\vfill

\subsection{10.  Exercise 7.23} 

\textbf{Question: }Show that if $n > 2$ then a quadratic residue mod (n) cannot also be a primitive root mod (n).  

\textbf{Solution: }

By definition, for $n > 2$ the set of quadratic residues $Q_n$ in $Z_n$ is a proper subgroup of the set of units in $Z_n$, $U_n$.  We also know that if a is in the set of quadratic residues then so are all powers of a.  Recall that an element a is a primitive root if every number coprime to it is congruent to a power of g modulo a.  Using the fact that the set of quadratic residues is strictly smaller than the set of units and the fact that all powers of a are in the set of quadratic residues for a quadratic residue, we know that there exists some elements of $U_n$ that are not a power of a and thus a cannot be a primitive root mod (n).  

\vfill

\end{spacing}
\end{document}
