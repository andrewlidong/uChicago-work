\documentclass{article}
% Change "article" to "report" to get rid of page number on title page
\usepackage{amsmath,amsfonts,amsthm,amssymb}
\usepackage[all]{xy}
\usepackage{enumerate,verbatim}
\usepackage{setspace}
\usepackage{Tabbing}
\usepackage{fancyhdr}
\usepackage{lastpage}
\usepackage{extramarks}
\usepackage{chngpage}
\usepackage{soul,color}
\usepackage{graphicx,float,wrapfig}

% In case you need to adjust margins:
\topmargin=-0.45in      %
\evensidemargin=0in     %
\oddsidemargin=0in      %
\textwidth=6.5in        %
\textheight=9.0in       %
\headsep=0.25in         %

% Homework Specific Information
\newcommand{\hmwkTitle}{Notes}
\newcommand{\hmwkClass}{Abstract Algebra}
\newcommand{\hmwkClassInstructor}{Rohit Nagpal}
\newcommand{\hmwkAuthorName}{Andrew Dong}

% Setup the header and footer
\pagestyle{fancy}                                                       %
\lhead{\hmwkAuthorName}                                                 %
\chead{\hmwkClass\ (\hmwkClassInstructor : \hmwkTitle)}  %
\rhead{\firstxmark}                                                     %
\lfoot{\lastxmark}                                                      %
\cfoot{}                                                                %
\rfoot{Page\ \thepage\ of\ \pageref{LastPage}}                          %
\renewcommand\headrulewidth{0.4pt}                                      %
\renewcommand\footrulewidth{0.4pt}                                      %

% This is used to trace down (pin point) problems
% in latexing a document:
%\tracingall

%%%%%%%%%%%%%%%%%%%%%%%%%%%%%%%%%%%%%%%%%%%%%%%%%%%%%%%%%%%%%
% Some tools
\newcommand{\enterProblemHeader}[1]{\nobreak\extramarks{#1}{#1 continued on next page\ldots}\nobreak%
                                    \nobreak\extramarks{#1 (continued)}{#1 continued on next page\ldots}\nobreak}%
\newcommand{\exitProblemHeader}[1]{\nobreak\extramarks{#1 (continued)}{#1 continued on next page\ldots}\nobreak%
                                   \nobreak\extramarks{#1}{}\nobreak}%

\newlength{\labelLength}
\newcommand{\labelAnswer}[2]
  {\settowidth{\labelLength}{#1}%
   \addtolength{\labelLength}{0.25in}%
   \changetext{}{-\labelLength}{}{}{}%
   \noindent\fbox{\begin{minipage}[c]{\columnwidth}#2\end{minipage}}%
   \marginpar{\fbox{#1}}%

   % We put the blank space above in order to make sure this
   % \marginpar gets correctly placed.
   \changetext{}{+\labelLength}{}{}{}}%

\setcounter{secnumdepth}{0}
\newcommand{\homeworkProblemName}{}%
\newcounter{homeworkProblemCounter}%
\newenvironment{homeworkProblem}[1][Problem \arabic{homeworkProblemCounter}]%
  {\stepcounter{homeworkProblemCounter}%
   \renewcommand{\homeworkProblemName}{#1}%
   \section{\homeworkProblemName}%
   \enterProblemHeader{\homeworkProblemName}}%
  {\exitProblemHeader{\homeworkProblemName}}%

\newcommand{\problemAnswer}[1]
  {\noindent\fbox{\begin{minipage}[c]{\columnwidth}#1\end{minipage}}}%

\newcommand{\problemLAnswer}[1]
  {\labelAnswer{\homeworkProblemName}{#1}}

\newcommand{\homeworkSectionName}{}%
\newlength{\homeworkSectionLabelLength}{}%
\newenvironment{homeworkSection}[1]%
  {% We put this space here to make sure we're not connected to the above.
   % Otherwise the changetext can do funny things to the other margin

   \renewcommand{\homeworkSectionName}{#1}%
   \settowidth{\homeworkSectionLabelLength}{\homeworkSectionName}%
   \addtolength{\homeworkSectionLabelLength}{0.25in}%
   \changetext{}{-\homeworkSectionLabelLength}{}{}{}%
   \subsection{\homeworkSectionName}%
   \enterProblemHeader{\homeworkProblemName\ [\homeworkSectionName]}}%
  {\enterProblemHeader{\homeworkProblemName}%

   % We put the blank space above in order to make sure this margin
   % change doesn't happen too soon (otherwise \sectionAnswer's can
   % get ugly about their \marginpar placement.
   \changetext{}{+\homeworkSectionLabelLength}{}{}{}}%

\newcommand{\sectionAnswer}[1]
  {% We put this space here to make sure we're disconnected from the previous
   % passage

   \noindent\fbox{\begin{minipage}[c]{\columnwidth}#1\end{minipage}}%
   \enterProblemHeader{\homeworkProblemName}\exitProblemHeader{\homeworkProblemName}%
   \marginpar{\fbox{\homeworkSectionName}}%

   % We put the blank space above in order to make sure this
   % \marginpar gets correctly placed.
   }%

%%%%%%%%%%%%%%%%%%%%%%%%%%%%%%%%%%%%%%%%%%%%%%%%%%%%%%%%%%%%%


%%%%%%%%%%%%%%%%%%%%%%%%%%%%%%%%%%%%%%%%%%%%%%%%%%%%%%%%%%%%%
% Make title
\title{\vspace{2in}\textmd{\textbf{\hmwkClass:\ \hmwkTitle}}\\\normalsize\vspace{0.1in}\vspace{0.1in}\large{\textit{\hmwkClassInstructor}}\vspace{3in}}
\date{}
\author{\textbf{\hmwkAuthorName}}
%%%%%%%%%%%%%%%%%%%%%%%%%%%%%%%%%%%%%%%%%%%%%%%%%%%%%%%%%%%%%
\newcommand{\inv}{^{-1}}
\begin{document}
\begin{spacing}{1.1}
\newpage
% Uncomment the \tableofcontents and \newpage lines to get a Contents page
% Uncomment the \setcounter line as well if you do NOT want subsections
%       listed in Contents
%\setcounter{tocdepth}{1}
%\tableofcontents
%\newpage

% When problems are long, it may be desirable to put a \newpage or a
% \clearpage before each homeworkProblem environment

\section{Contents}
\subsection{Groups and Subgroups}
1. Introduction and Examples
\\ 2. Binary Operations
\\ 3. Isomorphic Binary Structures
\\ 4. Groups
\\ 5. Subgroups
\\ 6. Cyclic Groups
\\ 7. Generating Sets and Cayley Digraphs

\subsection{Permutations, Cosets and Direct Products}
8. Groups of Permutations
\\ 9. Orbits, Cycles and the Alternating Groups
\\ 10. Cosets and the Theorem of Lagrange
\\ 11. Direct Products and Finitely Generated Abelian Groups
\\ 12. Plane Isometries

\subsection{Homomorphisms and Factor Groups}
13. Homomorphisms
\\ 14. Factor Groups
\\ 15. Factor-Group Computations and Simple Groups
\\ 16. Group Action on a Set
\\ 17. Applications of G-Sets to Counting

\subsection{Rings and Fields}
18. Rings and Fields
\\ 19. Integral Domains
\\ 20. Fermat's and Euler's Theorems
\\ 21. The Field of Quotients of an Integral Domain
\\ 22. Rings of Polynomials
\\ 23. Factorization of Polynomials over a Field
\\ 24. Noncommutative Examples
\\ 25. Ordered Rings and Fields

\subsection{Ideals and Factor Rings}
26. Homomorphisms and Factor Rings
\\ 27. Prime and Maximal Ideals
\\ 28. Grobner Bases for Ideals

\subsection{Extension Fields}
29. Introduction to Extension Fields
\\ 30.   Vector Spaces
\\ 31. Algebraic Extensions
\\ 32. Geometric Constructions
\\ 33. Finite Fields

\subsection{Advanced Group Theory}
34. Isomorphism Theorems
\\ 35. Series of Groups
\\ 36. Sylow Theorems
\\ 37. Applications of the Sylow Theory
\\ 38. Free Abelian Groups
\\ 39. Free Groups
\\ 40. Group Presentations

\subsection{Groups in Topology}
41. Simplicial Complexes and Homology Groups
\\ 42. Computations of Homology Groups
\\ 43. More Homology Computations and Applications
\\ 44. Homological Algebra

\subsection{Factorization}
45. Unique Factorization Domains
\\ 46. Euclidean Domains
\\ 47. Gaussian Integers and Multiplicative Norms

\subsection{Automorphisms and Galois Theory}
48. Automorphisms of Fields
\\ 49. The Isomorphism Extension Theorem
\\ 50. Splitting Fields
\\ 51. Seperable Extensions
\\ 52. Totally Inseperable Extensions
\\ 53. Galois Theory
\\ 54. Illustrations of Galois Theory
\\ 55. Cyclotomic Extensions
\\ 56. Insolvability of the Quintic

\subsection{Matrix Algebra}

\clearpage


\section{Basic Concepts}
\subsection{Probability and Relative Frequency}
P(A) = $\frac{(N(A))}{N}$ 
\\ where N is the total number of ourcomes of the experiment and N(A) is the number of outcomes leading to the occurrence of the event A

The ratio $\frac{n(A)}{n}$ is called the relative frequency of the event A.  It turns out that the relative frequencies $\frac{n(A)}{n}$ are virtually the same for large n, clustering about some constant
\\ $P(A)\sim \frac{n(A)}{n}$



etc

\end{spacing}
\end{document}
