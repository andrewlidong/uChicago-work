\documentclass[11pt]{article}
\usepackage{amsmath,amsthm,amssymb,fancyhdr,graphicx}
\usepackage[margin=1in]{geometry}

\newtheorem{thm}{Theorem}[section]
\newtheorem{lem}[thm]{Lemma}
\newtheorem{prop}[thm]{Proposition}
\newtheorem{cor}[thm]{Corollary}

\newenvironment{defn}[1][Definition]{\begin{trivlist}
\item[\hskip \labelsep {\bfseries #1}]}{\end{trivlist}}
\newenvironment{ex}[1][Examples]{\begin{trivlist}
\item[\hskip \labelsep {\bfseries #1}]}{\end{trivlist}}
\newenvironment{rem}[1][Remark]{\begin{trivlist}
\item[\hskip \labelsep {\bfseries #1}]}{\end{trivlist}}
\newenvironment{app}[1][Application]{\begin{trivlist}
\item[\hskip \labelsep {\bfseries #1}]}{\end{trivlist}}


 % New Commands
 % The rough rule for defining these new commands is as follows. If the original command is not a compound word, then take the first two letters of that command. If the command is a compound word, then take the first letter of each component word. 
 % special text
\newcommand{\bb}[1]{\mathbb{#1}}
\newcommand{\ca}[1]{\mathcal{#1}}
\newcommand{\mb}[1]{\mbox{#1}}
\newcommand{\un}[1]{\underline{#1}}
\newcommand{\ov}[1]{\overline{#1}}
 % cdots, ldots, times, arrows, set relations
\newcommand{\cd}{\cdots}
\newcommand{\Lra}{\Leftrightarrow}
\newcommand{\ld}{\ldots}
\newcommand{\ti}{\times}
\newcommand{\ra}{\rightarrow}
\newcommand{\Ra}{\Rightarrow}
\newcommand{\Lr}{\Leftrightarrow}
\newcommand{\hra}{\hookrightarrow}
\newcommand{\sbs}{\subset}
\newcommand{\sps}{\supset}
\newcommand{\sbse}{\subseteq}
\newcommand{\spse}{\supseteq}
\newcommand{\bs}{\backslash}
\newcommand{\ci}{\circ}
\newcommand{\es}{\emptyset}
\newcommand{\pa}{\partial}
\newcommand{\wh}[1]{\widehat{#1}}
\newcommand{\lan}{\langle}
\newcommand{\ran}{\rangle}
 % Greek letters
\newcommand{\al}{\alpha}
\newcommand{\be}{\beta}
\newcommand{\ga}{\gamma}
\newcommand{\de}{\delta}
\newcommand{\ep}{\epsilon}
\newcommand{\vp}{\varphi}
\newcommand{\Om}{\Omega}
\newcommand{\De}{\Delta}
\newcommand{\si}{\sigma}
\newcommand{\Si}{\Sigma}
\newcommand{\la}{\lambda}
 % Sums!
\newcommand{\su}[2]{\sum_{#1}^{#2}}
\newcommand{\bca}[2]{\bigcap_{#1}^{#2}}
\newcommand{\bcu}[2]{\bigcup_{#1}^{#2}}
\newcommand{\fr}[2]{\frac{#1}{#2}}
 % Number Systems
\newcommand{\re}{\mathbb{R}}
\newcommand{\q}{\mathbb{Q}}
\newcommand{\co}{\mathbb{C}}
\newcommand{\z}{\mathbb{Z}}
\newcommand{\zp}{\mathbb{Z}_+}
\newcommand{\na}{\mathbb N}

\setlength{\parindent}{0pt} 
\setlength{\parskip}{1.25ex}

\pagestyle{fancy}
\headheight 14pt
\lhead{Vaidehee}

\begin{document}

\begin{center}
Lecture 2

Jan 5, 2011
\end{center}


{\bf Correction:}


An algebraic set $X$ in $\co^n$ is said to be {\bf irreducible} if it satisfies the two conditions:
\begin{enumerate}
\item $X$ cannot be written as a nontrivial union of two algebraic sets in $\co^n$.
\item $X\neq \es$.
\end{enumerate}

An {\bf affine variety in $\co^n$} is an irreducible algebraic set in $\co^n$.

{\bf Remark:} For an algebraic set $X$ in $\co^n$ with co-ordinate ring $A$, $X$ is irreducible $\Lr$ $A$ is an integral domain. ( $0$ ring is not an integral domain.)

{\bf Examples:}
\begin{enumerate}
\item $X = \{(x,y) \in \co^2 : xy = 0\}$ is an algebraic set, but not an affine variety. Note that the co-ordinate ring $A$ of $X$ is not an integral domain  ($x$ and $y$ are nonzero elements whose product is zero.)
\item $Y= \{(x:y:z)\in {\bf P ^2}(\co): y^2z = x^3+z^3 \}$ is a projective algebraic variety, not an affine algebraic variety. Define $X = \{(x,y) \in \co^2 : y^2 = x^3 + 1\}$. $X$ is an affine algebraic variety, we can identify it as a subset of $Y$ via the map $(x,y) \ra (x:y:1)$. (Under this identification) $Y-X = \{(0:1:0)\}$.

\end{enumerate}

{\bf Geometric meanings of ideals}

Let $X$ be a topological space.
$A=$ $\cal{C}$ $ (X)= \{f:X\ra \co$  continuous $\}$.
For a subset $S$ of $X$, define $I(S)= \{ f\in A: f(p)$ for all $ p\in S\}$. This is an ideal of $A$.

For a subset $E$ of $A$, define $V(E)=\{p \in X: f(p)=0$ for all $f\in E\}$. This is a closed subset of $X$.

[{\bf Justification:} It is the arbitrary intersection of closed sets $V(\{f\})$, $f\in E$.]

{\bf Proposition:} $I(S) = I(\overline S )$ for a subset $S$ of $X$.

$V(E) = V(J)$ for a subset $E$ of $A$ where $J$ is the ideal generated by $E$.

{\bf Remark:} If $X$ is a finite set with discrete toplogy, then there is a bijection between (closed) subsets of $X$ and ideals of $A$ via operations $I$ and $V$. 


In general, the following properties hold:
\begin{enumerate}
\item This correspondence is inclusion reversing:

For subsets $E_1\sbs E_2$ of $A$, $V(E_1) \sps V(E_2)$. For subsets $S_1\sbs S_2$ of $X$, $I(S_1)\sps I(S_2)$.
\item $I(X)=(0)$, $I(\es)=A$, $V(A)=\es$, $V((0))=X$.
\item For $p\in X, I(\{p\})=\{f \in A : f(p)=0\}$ is a maximal ideal of $A$.

[The evaluation map $ev_p$ is a surjective ring homomorphism from $A$ to the field $\co$ with kernel $I(\{p\})$.]
\end{enumerate}


{\bf Proposition 1.2.1:} For a compact Hausdorff space $X$ we have
\begin{enumerate}
\item For $S\sbs X$, $V(I(S))=\overline S$.
\item $S\sbs X$ is closed $\Lr$ $S=V(I)$ for some ideal $I$ of $A$.
\item If $I$ is an ideal of $A$ such that $V(I)=\es$ then $I=A$.
\end{enumerate}
{\bf Theorem 1.1.4} For a compact Hausdorff space $X$, $p\ra I(\{p\})$ is a bijection between the sets $X$ and $max(A)$.

{\bf Proof:} Surjectivity - Let $I\in max(A)$. By 3, $V(I)\neq \es$. Choose $p \in V(I)$.  $I\sbs I(\{p\})$. By maximality of $I$, $I=I(\{p\})$. 

Injectivity- Any two distinct points of $X$ are separated by a continuous complex valued function.

{\bf Love story between Algebra and Topology}

Can recover topology of $X$ using $A$:

There is a bijection between closed sets of $X$ and $\{m\in max(A): m\sps I\}$ for some ideal $I$ of $A$.














\subsection{Overview: Basic Measure Theory}

\begin{defn}
{\bf algebra}: Set $X$, $\cal{A}\sbs 2^X$; $\cal{A}\neq\emptyset$; $\cal{A}$ closed under union and and complements. 
\end{defn}

\begin{defn}
$\si$-algebra
\end{defn}

\begin{defn}
{\bf pre-measure} $m:\cal{A}\ra\re_+$ is a measure on $\cal{A}$ such that
\begin{enumerate}
\item $m(\es)=0$.
\item measure of finitely many mutually disjoint sets is the sum of individual measures (finitely additive)
\item if $A\in\cal{A}$ and $A\sbs\bcu{j=1}{\infty}A_j$, then $m(A)\le\su{j=1}{\infty}m(A_j)$. This is called countably sub-additive. 
\end{enumerate}
\end{defn}

We will construct measures from pre-measures by making outer measures and restricting to measurable sets. 

Measure theory has no topology, but the plane does, so it'd be silly not to use it.

\begin{defn}
$\lambda:2^X\ra[0,\infty]$ is an {\bf outer measure} if 
\begin{enumerate}
\item $\la(\es)=0$
\item $\la$ is $\si$-subadditive
\end{enumerate}
\end{defn}

\begin{defn}
Define $C_{\la}$ to be the set of measurable sets; it's contained in $2^X$. 

$E\in C_{\la}$ if for all $S\sbs X$, $\la(S)=\la(S\cap E)+\la(S\cap E^c)$.

\end{defn}

$C$ stands for Caratheodory.

\begin{defn}
$\mu:\Sigma\ra\re_+$ is a {\bf measure} if $\Sigma$ is a $\si$-algebra and
\begin{enumerate}
\item $\mu(\es)=0$
\item $\mu$ is countably additive.
\end{enumerate}
\end{defn}

\begin{defn}
$\la$ is {\bf complete} if subsets of null sets are null.
\end{defn}

\begin{prop}
Let $\la$ be an outer measure on $X$. Then 
\begin{enumerate}
\item[(i)] $C_{\la}$ is a $\si$-algebra
\item[(ii)] $\la|_{C_{\la}}$ is a measure. 
\item[(iii)] $\la|_{C_{\la}}$ is complete. (if $A\sbs E, E\in C_{\la}$, then $\la(E)=0\Ra A\in C_{\la}$. 
\end{enumerate}

\end{prop}
\begin{proof}
It is enough to check that $\la(S)\ge \la(S\cap E)+\la(S\bs E)$ for all $S$. (the other inequality follows from finite sub-additivity. 

Claim: if $E,F\in C_{\la}$, then $E\cup F\in C_{\la}$. 

Indeed 
\[\la(S)=\la(S\cap E)+\la(S\bs E)\ge\la(S\cap E)+\la((S\bs E)\cap F)+\la((S\bs E)\bs F)\]

So we do have an algebra...

We claim that $\bcu{j=1}{\infty} E_j\in C_{\la}$. (wlog, $E_j\cap E_k=\es$).

\end{proof}

. . . 

Let $m:\cal{A}$$\ra[0,\infty]$ a pre-measure on an algebra on a set $X$. 

\begin{defn}
Define $\la_m(S)=\inf\su{j=1}{\infty}m(A_j)$, where $S\sbs\bcu{j=1}{\infty}A_j$, $A_j\in\cal{A}$. Then $\la_m:2^x\ra[0,\infty]$.
\end{defn}

\begin{prop}
\begin{enumerate}
\item $\la_m$ is an outer measure
\item $A\sbs C_{\la_m}$
\item $\la_m|_{\cal{A}}=m$. 
\end{enumerate}
\end{prop}

\begin{proof}
in 1, we approximate each element $S_j$ in a covering of $S$. 

properties 2 and 3 are {\bf homeworks}. 3 uses $\si$-subadditivity. 

\end{proof}

\begin{defn}
$m$ is {\bf $\si$-finite} if there is a countable collection of measurable sets covering $X$ each of which has finite mass.
\end{defn}

\begin{prop}
If $m$ is $\si$-finite, then if $\mu_1$ is an extension of $m$ to a measure on $\Sigma$, a $\si$-algebra, $\Sigma\sps\cal{A}$, then $\mu_1|_{\si(\cal{A})}=\la_m|_{\si(\cal{A})}$.
\end{prop}
\begin{proof}
{\bf Homework}
\end{proof}

\begin{defn}
the sigma algebra generated by $A$ is denoted $\si(A)$. 
\end{defn}

\subsection{}

$(X,\Si)$ is called a {\bf measurable space}, and we'll talk about function $X\ra\re\cup\{\pm\infty\}$.

$f$ is {\bf measurable} if . . . 

If $g$ is continuous and $f$ is measurable, then $f\circ g$ is measurable.

$g$ continuous, means $g^{-1}(B)\sbs\cal{B}$, where $\cal{B}=\si(\cal{O})$. $f$ returns borelians to measurable.

\begin{prop}
Assume $f=(f_1,\ld,f_n):X\ra\re$ are measurable. Let $A\sbs\re^n$ and $A\in\cal{B}$$(\re^n)$. $f^{-1}(A)=\{x\mid (f_1(x),\ld,f_n(x))\in A\}$. 

$f^{-1}(A)\in\Sigma$.
\end{prop}

\section{Day 2}

\subsection{Convergence}

\begin{defn}
Let $\la:2^{\re^n}\ra[0,\infty]$ be an outer-measure. Then $\la$ is {\bf Borel regular} if 1) $\cal{B}$$(\re^n)\sbs C_{\la}$ and 2) for any $S\sbs \re^n$, there exists $B\in\cal{B}$$(\re^n)$  we have $\la(B)=\la(S)$. 
\end{defn}

\begin{defn}
We say $\mu$ is {\bf Radon} if $\mu$ is Borel-regular and for all  $K\sbs\sbs\re^n$ (means compact)
\end{defn}

\begin{thm}
$\la$ Randon on $\re^n$. 

1) for all $S\sbs\re^n$ $\la(S)=\inf\la(U)$ such that $S\bs U$ and $U$ is open.

2) for all $A\in C_{\la}$, $\la(A)=\sup\la(K)$ where $K\sbs\sbs A$ is compact.
\end{thm}

\subsection{Convergence of Functions}

$f_n:X\ra\re$ measurable functions, where $(X,\Si,\mu)$ is a measure space. 

\begin{defn}
$f_n\ra f$ in measure if for all $\ep>0$, \[\lim_{n\ra\infty}\mu(\{x: |f_n(x)-f(x)|\ge\ep\})=0\]
\end{defn}

\begin{defn}
$f_n\ra f$ $\mu$-a.e.\ if $\mu(\{f_n $ does not converge to $f\})=0$. 
\end{defn}

\begin{prop}
If $\mu(x)$ is finite then $f_n\ra f$ a.e. implies $f_n\ra f$ in measure. 
\end{prop}

\begin{prop}
If $f_n\ra f$, then there is a subsequence $f_{n_k}\ra f$ almost everywhere.
\end{prop}
{\bf exercise:}

{\it Hint}: Pick a sequence of $\ep$ say $1/n$. . .\\
find $N_m$ such that $\mu(\{\mid |f_j-f|\ge 1/m\}\le 2^{-m}$ for all $j\ge N_m$. $\fr{1}{m}$ is our $\ep$. 

wlog, $N_1<N_2<\cd$. 

Claim: $f_{N_m}\ra f$ almost everywhere. 

\subsection{Something else}

\begin{thm}[Egorov]
Let $\mu$ be a Borel regular measure on a compact topological space $X$ (or $\mu$ Radon in $\re^n$). Let $A$ with $\mu(A)<\infty$ and $f_n\ra f$ a.e.\ on $A$. Then for every $\ep>0$, there exists $K\sbs\sbs A$, $\mu(A\bs K)\le \ep$, $f_n\ra f$ uniformly on $K$.
\end{thm}
\begin{proof}
We use the lemma below.

Take $H_m$ such that $\mu(H_m)<\ep/(2^{m+1})$ so $|f(x)-f_n(x)|\le\fr{1}{2^m}$ for all $n\ge M_m$. We approximate these guys by compact because of regularity.

$H=\bcu{m}{}H-m$. $\mu(H)\le \ep/2$. $\lim f_n(x)=f(x)$ uniformly on $A\bs H$. 

$|f_n(x)-f(x)|\le \fr{1}{2^m}\le \ep$ for all $n\ge N_m$. 

Now take $K\sbs\sbs A\bs H$ such that $\mu((A\bs H)\bs K)\le \ep/2$. This is the only place where we use regularity.

\end{proof}


(Almost everywhere convergence is almost as good as uniform convergence if you are willing to throw away a small something.)

\begin{lem}
$\mu$ is. Then for all $\ep,\de$, there is $N$ such that $f_n\ra f$ a.e\ on $A$, and $H\sbs A$, such that $\mu(H)\le\ep$. Then $|f_n(x)-f(x)|\le \de$ for all $n\ge N$ and for all $x\in A\bs H$. 
\end{lem}
\begin{proof}
See, paper notes.
\end{proof}

\begin{thm}[Lusin]
Take $\mu$ (Radon) on an arbitrary topological space. Let $A$ be a measurable and $0<\mu(A)\le\infty$. Let $f$ be measurable. Then for all $\ep$, there exists $K\sbs\sbs A$ and $\mu(A\bs K)\le \ep$ and $f|_K$ is continuous. 
\end{thm}

\begin{lem}
Let $\mu$ be Radon, $A$ with $\mu(A)<\infty$. Let $f$ be a simple function. 
Then for all $\ep$, there exists $K\sbs\sbs A$ with $\mu(A\bs K)\le \ep$ and $f|_K$ is continuous. 
\end{lem}
\begin{proof}
This is trivial. Put $K_j\sbs\sbs A_j$ such that $\mu(A_j\bs K_j)\le \ep/N$. Put $K=\cup K_j$. Let $f|_K=g(x)=\sum c_j1_{K_j}$. It is continuous. 
\end{proof}

{\it Proof of Lusin}: 
\begin{proof}
There exists $f_n$ simple such that $f_n\ra f$ a.e.\ ({\bf exercise}). For all $m$ there exists $K_m\sbs\sbs A$ with $\mu(A\bs K_m)\le \ep/2^{m+1}$ so that $f_m|_{K_m}$ is continuous by our lemma. 

By Egorov there exists $L\sbs\sbs A$ such that $\mu(A\bs L)\le\ep/2$ and $f_n\ra f$ uniformly on $L$. 

Define $K=L\cap \bca{n=1}{\infty}K_n$. $f_n\ra f$ uniformly, $f_n$ is continuous on $K$. Therefore $f$ is continuous on $K$. 

$K$ is nonempty because $\ep<\mu(A)$. 
\end{proof}

\subsection{Integration}

\begin{defn}
$\mu$ complete on $X$, $f\ge 0$ a.e.\, then $\int f$ is . . . 
\end{defn}

{\bf Elementary Properties of integral}:
monotonic, scalars behave,

$f=0$ on $E$ $\mu$ a.e.\ then $\int_E fd\mu=0$.

$\mu(E)=0$ means $\int_E fd\mu=0$. Even if $f=\infty$ on $E$.

$\int_A fd\mu=\int_X1_Afd\mu$. 

$A\mapsto \int_A\phi d\mu$ is a measure for all $\phi$ simple.  

\begin{defn}
If $f\ge 0$ measurable and $\int fd\mu<\infty$, then we say $f$ is {\bf integrable}. 
\end{defn}

\subsection{Lebesgue Monotone}

\begin{thm}
$0\le f_n\le f_{n+1}$ measurable a.e. Let $f(x)=\lim f_n(x)$ a.e. Then $\int f=\lim_{n\ra\infty}\int f_n$. 
\end{thm}

\section{Day 3}

The following are equivalent. Let $\la$ be a measure on $[0,1]^n$. 
\begin{enumerate}
\item $A$ is measurable in $C_{\la}$. 

\item $A$ is inner regular
\item $AA$ is outer regular. 

\end{enumerate}

$n=1$: a non-measurable set. All lebesgue measurable sets are translation invariant...

Let $\q=\{r_0=0,r_1,\ld\}$. Define $x\sim y$ if $x-y\in\q$. For $p\in P$, there exists . . . 

$1=\su{n=1}{\infty}\la(P)$. 

\subsection{Markov Inequality}
\begin{thm}
$f:X\ra \re$ a measurable function on measurable $X$ and $f$ is integrable. Then \[\mu(x\in X\mid |f|\ge \la)\le \fr{1}{\la}\int|f|d\mu.\] 
\end{thm}


\begin{cor}[Beppoulevi]
Suppose that $0\le f_n$, $f_n\le f_{n+1}$. Assume that $\int f_nd\mu\le C$, some $C>0$ for all $n$. Then $\lim f_n=f$ exists a.e.\ and is finite. $\lim \int f_n=\int f<\infty$.  
\end{cor}
This is a corollary to Lebesgue dominated. 
\begin{proof}
By Lebesgue monotone, 

Let $\tilde f=\lim f_n(x)$. $\lim\int f_n(x)d\mu=\int\tilde fd\mu\le C$. 

From Markov inequality, $\mu(\{x\mid \tilde f(x)\ge M\})\le C/M$. Then $Z=\{x\mid \tilde f(x)=\infty\}=\bca{M=1}{\infty}\{x\mid \tilde f(x)\ge M\}$. 
\[\mu(Z)=\lim_{M\ra\infty}\mu(\{x\mid \tilde f(x)\ge M\})=0\]

Then the limit exists almost everywhere, $\tilde f=f$. 
\end{proof}

{\bf Fatou}

\begin{thm}[Fatou's Lemma]
Let $f_n\ge 0$ measurable. $f_n:X\ra \re_+$. Then $\int\lim\inf f_n\le \lim\inf\int f_nd\mu$. 
\end{thm}
\begin{proof}
This follows from Lebesgue monotone. 
Take $g_n=\inf_{k\ge n} f_k(x)$. By definition of $\lim\inf$, \[\lim\inf_{n\ra\infty}f_n(x)=\lim_{n\ra\infty}g_n=\sup_{n\ra\infty}g_n(x).\]

$0\le g_n(x)\le g_{n+1}(x)$. Then we have a non-decreasing sequence of non-negative function. By Lebesgue monotone convergence, \[\lim_{n\ra\infty}\int g_nd\mu=\int\lim g_n(x)d\mu=\int\lim\inf f_n(x)d\mu.\]

OTOH, $g_n(x)=\inf_{k\ge n}f_k(x)\le f_n(x)$, so $\int g_n(x)d\mu\le\int f_nd\mu$. 

Then \[\lim_{n\ra\infty}\int g_n(x)d\mu\le\lim\inf_{n\ra\infty}\int f_n(x)d\mu.\]
\end{proof}

\begin{thm}[Lebesgue Dominated]
Let $f_n$ be a sequence of measurable functions. Assume there exists $g\ge 0\in L^1(d\mu)$ (i.e.\ $\int gd\mu<\infty)$ such that $|f_n(x)|\le g(x)$ almost everywhere. Assume $f_n(x)\ra f(x)$ a.e. Then \[\lim_{n\ra\infty}\int f_n(x)d\mu=\int f(x)d\mu.\]
\end{thm}
\begin{proof}
(Triangle inequality and passing through the limit). 

$|f_n-f|\le 2g$. Then $2g-|f_n-f|\ge 0$. 

From Fatou, \[\int\lim\inf_{n\ra\infty}(2g-|f_n-f|)d\mu\le\lim\inf_{n\ra\infty}\int(2g-|f-f_n|)\]

The left hand side is $\int 2gd\mu$. The right hand side is $\int 2g+\lim\inf\int(-|f_n-f|)$. So
\[\int 2g\le \int2g-\lim\sup\int|f_n-f|d\mu.\]

Then $\lim\sup\int|f_n-f|d\mu\le 0$. 

Hence we proved that \[\lim_{n\ra\infty}\int |f_n-f|d\mu=0.\] We also have \[\left|\int f_n-\int f\right|\le\int|f_n-f|d\mu\]
\end{proof}

$|\int f|\le \int |f|$ because $\int f_+-\int f_-\le \int( f_++f_-)$, which you should check.

{\bf Homework}: Show that if $f\ge 0, g\ge 0$, then $\int(f+g)=\int fd\mu+\int gd\mu$. 

\subsection{$L^p$ spaces}

\begin{defn} $L^p$ space.\end{defn}

\begin{lem}[Young]
$ab\le \fr{a^p}{p}+\fr{b^q}{q}$. $\fr{1}{p}+\fr{1}{q}=1$. $p\ge1$. 

Hint $x=ab^{1-q}$. Want $x<1/q+a^pb^{-q}/p$. $x^p$

$x\le 1/q+1/p\cdot x^p$. 

differentiate once, differentiate twice to get inequality. 
\end{lem}

{\bf Holder Inequality}


{\bf Minkowski}: triangle inequality in $L^p$. 

{\bf something else} 

\begin{thm}Let $X$ be a topological space, $\mu$ Radon and $X$ $\si$-finite. Let $1\le p<\infty$. Then $C_0(X)$ is dense in $L^p(d\mu)$. 
\end{thm}

$f\in L^{\infty}$ if $f$ is bounded almost everywhere. $||f||_{L^{\infty}}=\inf_M\{\mu\{(x: |f(x)|\ge M\}=0\}$. 

\section{Day 4} 

\subsection{Approximation}

Let $(X,\Sigma,\mu)$ a space. Let $1\le p<\infty$ and $L^p(d\mu)=\{f:X\ra\re: f$ measurable, $\int|f|^pd\mu<\infty\}$. We have inequalities that organize this space. We have distance. 

$d(f,g)=||f-g||_{L^p}$.

\begin{thm}
If $X$ is $\si$-finite, then $S_F=\{f\in L^p:\mu(\{x\mid f(x)\neq 0\})<\infty\}$ is dense in $L^p$. 

So for all $f\in L^p$, there is a $g_n\in S_F$ with $||g_n-f||\ra 0$. 
\end{thm}
\begin{proof}
Let $A_n\sbs A_{n+1}$ with $X=\bcu{}{}A_n$ and $\mu(A_n)<\infty$. 

Let $f\in L^p$. Call $g_n=f\cdot 1_{A_n}$. . . 

Use Lebesgue dominated. 

\end{proof}

\begin{prop}[approximated by bounded?]
Let $X$ with $\mu(X)<\infty$. There exists $h\in L^p$, $h$ bounded, $||h-g||\le \ep/2$. 

\end{prop}

\begin{prop}
If $X$ is $\si$-finite, for all $f\in L^p$, there exists $s_n$ simple functions such that $s_n\ra f$, where $s_n$ are nonzero on a finite set and bounded. 
\end{prop}
\begin{proof}
Use previous proposition, to approximate $f$ by $h$ bounded. Then approximate $h$ by simple functions. 
\end{proof}

Take $X$ topological space and $\mu$ radon on $X$. Assume $X$ $\si$-finite or $X$ locally compact. 

\[C_0(X)=\{f:X\ra\re: f\mb{ continuous, compactly supported}\}\]

\begin{prop}
We claim that $C_0(X)$ is dense in $L^p$. 
\end{prop}
\begin{proof}
Approximate with simple functions (or could have used Lusin property). 

$f\in L^p$. 

Let $g$ be bounded and $\mu(g(x)\neq 0)<\infty$. By Lusin, there exists $K\sbs A$ compact. 

$||g-f||\le \ep/3$. $\mu(A\bs K)\le \ep^p/(3^p||g||_{l^{\infty}}^p$. 

So that $g|_K$ is continuous. 

Let $g|_K=h$ continuous and 
\[||h-g||^p=\int|h-g|^pd\mu=\int_A|h-g|^p=\int_K|g-g|^p+|\int_{A\bs K}|g|^p\le||g||^p_{l^{\infty}}\mu(A\bs K)\le \ep/3\]

Problem: $h$ is not continuous. We need to use step functions. 

wlog need to approximate $1_A$ for $\mu(A)<\infty)$. 

\end{proof}

\subsection{Completeness of $L^p$} 

$||\cdot||:L^p\ra\re_+$, where $||f||_{L^p}=0\Lra f=0$. 

$||f+g||\le ||f||+||g||$

$||f\la||=|\la|||f||$ for all $\la\in\co$. 

So we have a norm. 

Let $f_n$ be a cauchy sequence in $L^p$. 


In a metric space, if $\{f_n\}$ is a Cauchy sequence and if there exists $f$ and subsequence $\{f_{n_k}\}\ra f$, then $f_n\ra f$. 

There exists $n_1\le n_2\le\cd$ so that $||f_{n_{k+1}}-f_{n_k}||\le 2^{-k}$. 

$g_k=|f_{n_{k+1}}-f_{n_k}|$. 

$s_k=\su{j=1}{k}|f_{n_{j+1}}-f_{n_j}|=\su{j=1}{k}g_j$. 

$s_k\ra$, $s_k^p\ra$. $||s_k||_{L^p}\le 1$. 

$s=\lim s_k$. 

$\int s^p=\lim\int s_k^p\le 1$. 

then $s<\infty$ a.e

Then $f_{n_1}+f_{n_2}+f_{n_1}+\cd+f_{n_{k+1}}-f_{n_k}=f$ converges. 

$||\su{j=m}{\infty}f_{n_{j+1}}-f_{n_j}||\le\su{j\ge m}{}||g_{n_{j+1}}-g_n||\le 2^{-m+1}$

$||f_{n_1}-f||\le 2^{-n_1+1}\ra 0$. 

$L^p$ is {\bf Banach Space}

Scalar product axioms on complex vector space. This gives a pre-hilbert space.

Give norm $||f-g||=\sqrt{\langle f-g,f-g\rangle}$.

{\bf Schwarz ineq}: 
$|\langle f,g\rangle|\le ||f||||g||$. 

Proof: Note that $\langle f+zg,f+zg\rangle\ge 0$ for all $z\in\co$. Expand this expression, and choose a good $z$. 

$||f+g||\le ||f||+||g||$ follows from schwarz. 

{\bf Parallelogram Identity}: $||f+g||^2+||f-g||^2=2||f||^2+2||g||^2$. 

Equivalently:  $||\fr{f+g}{2}||^2+||\fr{f-g}{2}||^2=\fr{||f||^2+||g||^2}{2}$. 

{\bf exercise}: parallelogram identity and normed linear space gives a hilbert space. 


\begin{thm}
Let $H$ be a Hilbert space. Let $C\sbs H$ be closed and convex. Then there exists $x\in C$ such that \[||x||=\inf_{c\in C}||c||\]
\end{thm}

This is false in Banach spaces in general.
\begin{proof}
%Let $\al=\inf_{c\in C}||c||$. $\al$ exists because we take inf of a set bounded below by 0. Pick sequence $c_n$ such that $||c_n||\ra\al$. $\al\le||c_n||\le\al+\fr{1]{n}$

Take 

\end{proof}

\section{Day 5}

Hausdorff spaces important because limits are unique, compact sets are closed, can separate two functions . . . 

\begin{prop}[Urysohn's Lemma]

\end{prop}
Create a sequence of upper semi-continuous and a lower semi-continuous functions that converge to the same thing. 

\begin{thm}[Riesz Representation Theorem]
\end{thm}

\section{Day 6}

$H$ a Hilbert space, so $H$ has a scalar product $\langle f,g\rangle\in\co$, $H\ti H\ra\co$. We have a norm $||f||=(\lan f,f\ran)^{1/2}$. 

Convergence $||f-f_n||\ra 0$ means $f_n\ra f$ in $H$. 

We proved, if $C\sbs H$ and closed (in the topology given by our norm) and convex, then there exists $c\in C$ such that $||c||=\inf_{x\in C}||x||$. Remember in the proof, we took a sequence that goes to $c$, and we used convexity to take midpoints to test. Midpoints must be farther from 0. Because of parallelogram identity, the midpoints approximate well. . .

\begin{prop}
Let $M\sbs H$ be a closed linear subspace. For $x\in H$ there exists unique $p\in M$ and $q\in M^{\perp}=\{h\in H\mid \forall a\in A, \lan h,a\ran=0\}$ such that $x=p+q$, $||x^2||=||p||^2+||q||^2$. 

2) $x\ra p=Px$ is linear and $x\ra q=Qx$ is linear. 
\end{prop}

$L^2([0,1])$ and polynomials, $L^2(\re)$ and continuous functions with compact support. 

$A^{\perp}$ is a closed linear subspace: {\it closed}: $A^{\perp}=\bca{a\in A}{}\ker L_a$, $L_a:H\ra\co$ by $L_a(g)=\lan g,a\ran$. 

$L_a:H\ra\co$ is linear, continuous (continuity is boundedness; because linear, you just check at the origin and extend linearly). $|L_a(g)|=|\lan g,a\ran|\le ||a||\cdot||g||$. 

\begin{proof}
$C:=x+M$ is closed (translation is continuous). $C$ is convex. Therefore, by the theorem, let $q=\mb{arg min}_{c\in C}||c||$. $||q||=\min_{m\in M}||x+m||$. 

Claim: $q\in M^{\perp}$. Take any $w\in M$, $||w||=1$. 

$||q||\le ||q+zw||$ for all $z$. $(\star)$

Why? We know $||q||\le||x+m||$ and $q\in C$. Then $q=x+\tilde m$. So $||x+m||=||q-\tilde m+m||$. 

Next we square ($\star)$ and we open brackets... Then $\lan w,q\ran=0$. Then $q\in M^{\perp}$. 

Now we are done. $x=p+q$, where $p=x-q\in M$. Then $||x||^2=||p||^2+||q||^2$ because $p$ and $q$ are perp. 

Uniqueness, if $x=p'+q'$, then since we can take differences, $p=p'=q'-q\in M\cap M^{\perp}$. Then note that $y\perp y$ implies $||y||^2=0$. 

Linearity follows from uniqueness. 
\end{proof}

$P:H\ra M$ has the following properties: {\bf self-adjoint}, {\bf idempotent}. 

\begin{thm}[Riesz]
Let $L:H\ra\co$ be a linear continuous functional. Then there exists unique $h_L=h$ such that $L(f)=\lan f,h\ran$. 
\end{thm}
\begin{proof}
If $L=0$, then take $h=0$. 

If $L\neq 0$, take $\ker L=M$, closed and linear. If $L\neq 0$, then $M\subsetneq H$, so there exists $\tilde h\in M^{\perp}$. 

Idea: for any $f$ make something in the kernel.

$m_f=(L\tilde h)f-(Lf)\tilde h$. $Lh$ and $Lf$ are complex numbers. So for each $f$ we associate vector $m_f$. 

$Lm_f=0$, easy to check. Then $m_f\in\ker L=M$. 

Then $\tilde h\in M^{\perp}$, $m_f\in M$ implies $\lan \tilde h,m_f\ran=0$. 
\[0=(L\tilde h)\lan f,\tilde h\ran-(Lf)||\tilde h||^2.\] 
So \[Lf=\left\lan f, (\ov{L\tilde h})\fr{\tilde h}{||\tilde h||^2}\right\ran\]

Trivial, but check the uniqueness. 

\end{proof}

\subsection{Orthonormal Sets}

Hilbert spaces are generalizations of Euclidean spaces, preserving the idea of orthogonality. 

\begin{defn}
Let $A$ a nonempty set (healthy to think of as natural numbers). We say that a family $\{e_a\}_{a\in A}$ is an {\bf orthonormal family} (ON) if $\lan e_a,e_b\ran=\de_{ab}$. 
\end{defn}
Let $\{e_a\}_{a\in A}$ be ON. Let $F\sbs A$ be finite. For all $f$, let \[S_F(f)=\su{a\in F}{}c_a(f)e_a.\]
$c_a(f)$ is the {\bf Fourier coefficient} and $c_a(f)=\lan f,e_a\ran$. 

What does this do? This is the answer to an optimization problem.

\begin{prop}
\[S_F(f)=\mb{arg min}_{s=\su{a\in F}{}\la_ae_a}||s-f||\]
\end{prop}
\begin{proof}
Idea: Consider $\ca S_F=\{s\mid s=\su{a\in F}{}\ga_ae_a, \ga_a\in\co\}$, a linear subspace of $H$. 

Claim: $f-S_F(f)\perp\ca S_F$. 

$g\perp \ca S$ if and only if $\lan g,e_a\ran=0$ for all $a\in F$. 

\[\lan f-S_F(f),e_b\ran=\lan f,e_b\ran-\lan\su{a\in F}{}c_a(f)\lan e_a,e_b\ran=\lan f,e_b\ran-c_b(f)=0.\]
$||s-f||=||s-S_F(f)+S_f(f)-f||$, where $s-S_F(f)\in \ca S$ and $S_f(f)-f\in\ca S^{\perp}$. Then $||s-f||^2=||s-S_F(f)||^2+||S_F(f)-f||^2\ge 0$. 
\[||S_F(f)-f||\le ||s-f||,\]
for all $s\in\ca S$.
\end{proof}

\begin{prop}
Let $\{e_a\}$ be a ON family in $H$. There exists $B\sps A$, $\{e_b\}_{b\in B}\sps\{e_a\}_{a\in A}$ and $\{e_b\}$ is maximal ON. 
\end{prop}

This follows from Mr.\ Zorn. If you have partial order, there is a maximal chain. . . 

TFAE: 
\begin{itemize}
\item[(i)] $\{e_a\}$ is maximal ON
\item[(ii)] The linear span of $\{e_a\}$ is dense in $H$
\item[(iii)] for all $f$, $||f||^2=\su{a\in A}{}|c_a(f)|^2$. 
\item[(iv)] for all $f,g$ $\lan f,g\ran=\su{a\in A}{}c_a(f)\ov{c_a(g)}$. 
\end{itemize}

\[\su{a\in A}{}|c_a|^2=\sup_{F\sbs A, F\mb{finite}}\su{a\in F}{}|c_a|^2=\int_A|c_a|^2d\mu\] where $\mu$ is the counting measure. 

$\mu^*:2^A\ra[0,\infty]$ and $\mu(B)=\infty$ if $B\sbs A$ is not finite, and $\mu^*(B)=$ number of elements of $B$ otherwise. What are the measurable sets according to Cartheodory. Simple functions are functions with finite support. This is how we constructed the integral. 

Note (iv) is obtain from (iii) from the parallelogram identity. The proof for the rest is in the book.

\subsection{Trigonometric Series}

$\al\in[0,2\pi]$. 

$e_j(\al)=\exp(ij\al)$. $\lan f,g\ran=\fr{1}{2\pi}\int_0^{2\pi}f(\al)\ov{g(\al)}d\al$. 

The $e_j$ are ON $j\in\z$. 

$\hat f(j)=\lan f,e_j\ran=\fr{1}{2\pi}\int_0^{2\pi}f(\al)\exp(-ij\al)d\al$. 

Define $P_n$ projection $\{e_{-n},\ld,e_0,\ld,e_n\}$. 

$P_n(f)=\su{|j|\le n}{}\hat f(j)e_j$. 

In physical space, $(P_nf)(\al)=\fr{1}{2\pi}\int_0^{2\pi}D_n(\al-\be)f(\be)d\be$, where $D_n$ is the Dirichlet kernel and 
\[D_n(\al)=\fr{\sin(n+1/2)\al}{\sin(\al/2)}=\su{|j|\le n}{}\exp(ij\al)\]

$C_n(f)=\fr{1}{n+1}(P_0f+\cd+P_nf)$. Then \[C_n(f)=\su{|j|\le n}{}\left(1-\fr{|j|}{n+1}\right)\hat f(j)e_j\]
This is a hat. Compared to $P_n f$ which is a step. . .  $C_n(f)$ is smooth. . . 

$C_n(f)(\al)=\fr{1}{2\pi}\int_0^{2\pi}F_n(\al-\be)f(\be)d\be$, where $F_n$ is called the {\bf Fejer} kernel. 

\[F_n(\al)=\fr{1}{n=1}\left(\fr{\sin((n+1)\al/2)}{\sin(\al/2)}\right)^2=\su{|j|\le n}{}\left(1-\fr{|j|}{n+1}\right)e_j\]

This is a good kernel. $D_n$ is a non-good kernel.

Properties of a good kernel: \\
1) $F_n\ge 0$. \\
2) $\fr{1}{2\pi}\int_0^{2\pi}F_n(\al)d\al=1$.\\
3) for all $\de>0$, \[\lim_{n\ra\infty}\int_{\de}^{2\pi-\de}F_n(\be)d\be=0\]

\begin{thm}
Let $f\in C(\Pi)$ ($f:[0,2\pi]\ra\co$, $f(0)=f(2\pi)$ and $f\in C[0,2\pi]$). 

\[\lim_{n\ra\infty}||C_n(f)-f||_{C(\Pi)}=0\]

$||g||_{C(\Pi)}=\sup_{\al\in[0,2\pi]}|g(\al)|$. 
\end{thm}
\begin{proof}
\[f(\al)-(C_nf)(\al)=-\fr{1}{2\pi}\int_0^{2\pi}F_n(\ga)f(\al-\ga)d\ga+\fr{1}{2\pi}\int_0^{2\pi}F_n(\ga)f(\al)d\ga\]
For the last summand, we used property 2. 

$f(\al)-(C_nf)(\al)=\fr{1}{2\pi}\int_0^{2\pi}F_n(\ga)(f(\al)-f(\al-\ga))d\ga$. 

Let $\ep>0$. There exists $\de$ such that if $\ga\in[0,\de]$ or $[2\pi-\de,2\pi)$, then $|f(\al)-f(\al-\ga)|\le \ep/M$ for all $\al$ because $f$ is uniformly continuous (using fact that we're on a compact). 

Then $\fr{1}{2\pi}\int_0^{\de}+\fr{1}{2\pi}\int_{2\pi-\de}^{2\pi}\le \int F_n(\ga)\ep/Md\ga\le\ep/M$. 

The other piece $\int_{\de}^{2\pi-\de}F_n|f-f|\le 2||f||$


\end{proof}

As a corollary $\{e_j\}$ are dense in $L^2$. 

\section{Day 7}

\subsection{Basics of Banach Space}

A {\bf Banach space} is a vectors space $X$ over $\re$ or $\co$ with a norm $||\cdot||:X\ra[0,\infty)$ such that 

(i) $||x||=0$ if and only if $x=0$\\
(ii) $||x+y||\le ||x||+||y||$\\
(iii) $||\la x||=|\la|\cdot||x||$.

Then we get a topology. $B(x,r)=B_r(x)=\{y\in X: ||y-x||<r\}$. Then $X$ is a metric space. $d(x,y)=||x-y||$. We have the property that translation is continuous. 

A Banach space has the additional property that it is complete. 

{\bf Examples}
\begin{enumerate}
\item $L^p(d\mu)$, $1\le p\le \infty$. 

\item $\ell^p(\z)=\{x\mid s=(x_n)_{n\in\z}, x_n\in\co, \su{n=-\infty}{\infty}|x_n|^p<\infty\}$

\item $\ell^p(\na)$

\item $C(K)$ with norm, supremum of absolute value... 
\end{enumerate}

{\bf Exercise}: $T:X\ra Y$ linear between $X,Y$ Banach. TFAE:

1) $T$ is continuous\\
2) $T$ is continuous at 0\\
3) $T$ is bounded. 

\begin{defn}
Linear $T:X\ra Y$ is bounded if $||T||<\infty$, where $||T||=\sup_{||x||\le1}\{||Tx||_Y\}$. 
\end{defn}

{\bf Exercise}: $||T||=\sup_{||x||=1}||Tx||$. 

{\bf Exercise}: $||T||=\inf\{c>0: ||Tx||\le c||x||, \forall x\}$. 

{\bf Exercise}: $X,Y$ Banach gives $L(X,Y)=\{T:X\ra Y$ bounded$\}$ is Banach with $||\cdot||$. 

\begin{defn}
$S=\bcu{n\in\na}{}A_n$ such that int$\ov{A}=\es$ for all $n$ is called of {\bf first category}. 
\end{defn}

\begin{thm}[Baire Category]
Let $X$ be a complete metric space. Then $X$ is of second category.
\end{thm}
\begin{proof}
Assume not. $X=\bcu{}{}A_n$, $\ov{A_n}=F_n$, int$(F_n)=\es$. 

$\bca{}{}D_n=\es$, where $D_n=X\bs F_n$ is open and dense. 

Take $x_1\in D_1$ and $B_1=B(x_1,r_1)\sbs D_1$. There exists $x_2\in B_1\cap D_2$. Then there exists open $B_2$ with $x\in B_2 \sbs D_2\cap B_1$. 

$B_2=B(x_2,r_2)\sbs\ov{B_2}\sbs B_1\cap D_2$. $r_2\le r_1/2$. 

Then $B_{i+1}\sbs B_i\cap D_{i+1}$. diam$B_{i+1}\le\fr{2r_1}{2^{i+1}}$. 

We claim $\{x_i\}$ is Cauchy. $d(x_{i+1},x_i)\le$ diam$B_i=$ summable. (distance between consecutive terms is summable implies sequence is cauchy). 

Then for $j\ge i$, $x_j\in\ov{B_i}\sbs D_i$. Then $x\in D_i$ for all $i$, a contradiction because $\bca{}{} D_n=\es$. 


\end{proof}

\begin{thm}[Banach-Steinhaus]
Let $X$ banach and $Y$ normed. Let $T_{\al}: X\ra Y$ be a family (indexed over $A$) of bounded linear operators. \\
(i) if for all $x\in X$, there exists $C=C_x>0$ such that \[\sup_{\al\in A}||T_{\al}x||\le C_x<\infty\] then there exists $C$ such that $\sup_{\al\in A}||T_{\al}||\le C<\infty$. \\
(ii) if $\sup_{\al}||T_{\al}||=\infty$, then there exists dense $G_{\de}\sbs X$ such that for all $x\in G_{\de}$, $\sup_{\al}||T_{\al}x||=\infty$. 
\end{thm}
\begin{proof}
(i) $\phi(x):=\sup_{\al\in A}||T_{\al}(x)||$. $x\mapsto ||T_{\al}(x)||$ is continuous. $\phi(x)=$ sup of family of continuous implies $\phi(x)$ is lower semi-continuous. 

Therefore $F_n\{x: \phi(x)\le n\}$ are closed. $X$ is Banach and $X=\bcu{n}{}F_n$ (for each $x$ we have a $C_x$, so $\phi(x)\le C_x$. 

Now by Baire Category, there exists $n_0$ such that int($F_{n_0})\neq\es$. Then there exists $x\in F_{n_0}$ and $r>0$ such that $B(x_0,r)\sbs F_{n_0}$. Then for all $y$ with $||y-x_0||<r$, $\phi(y)\le n_0$. (``Painful but true.") 

{\it Scaling and Translation}: for all $\al$, $||T_{\al}(y)||\le n_0$, $||y-x_0||<r$. From here we translate to say $\sup_{||z||<1}||T_{\al}(z)||\le C$. 

Write $y=rz+x_0$, with $||z||<1$. Then $||y-x_0||=||rz||=r||z||<r$ okay. 

$T_{\al}(y)=rT_{\al}(z)+T_{\al}(x_0)$. $T_{\al}(z)=\fr{1}{r}(T_{\al}(y)-T_{\al}(x_0))$. 

Then $||T_{\al}(z)||=\fr{1}{r}||T_{\al}(y)-T_{\al}(x_0)||\le \fr{2n_0}{r}$. Does not depend on $\al$. 

(ii) The set $F_n=\{x:||T_{\al}(x)||\le n,\forall\al\}=\{x\mid \phi(x)\le n\}$ is closed. 

No $F_n$ can have interior (by previous point's proof). Then $\bca{}{}X\bs F_n$ is dense. 

\end{proof}

\begin{thm}[Open Mapping]
$T:X\ra Y$ with $X,Y$ Banach, $T\in L(X,Y)$. Assume $T$ is onto. Then $T$ is open. 
\end{thm}
\begin{proof}
Note that $T$ is open if and only if there exists $r>0$ with $T(B(0,1))\sps B(0,r)$. 

By scaling and translation, this implies that the image of any ball contains a set that is open. 

$TB_X(x,\rho)\sps Tx+B_Y(0,\rho r)=B_{Tx}(\rho r)=\{Tx+y:||y||<\rho r\}$.

$TU$ where $U$ is open. Claim $TU$ is open. Take $x\in U$, there exists $\rho$ such that $B(x,\rho)\sbs U$. We know that $B(Tx,\rho r)\sbs T(B(x,\rho))$. Then $B(Tx,\rho r)\sbs TU$. 

Now we actually prove $T(B(0,1))\sps B(0,r)$. 

Step 1: there exists $r$ such that $\ov{T(B(0,1))}\sps B(0,\de)$. 

$\bcu{n=1}{\infty}T(B(0,1))=Y$. (Since surjective, everyone in image, so everyone in image of some ball.) Then there exists $n_0$ such that $\ov{T(B(0,n_0))}$ has interior. By rescaling $\ov{TB(0,1)}$ has interior. 

Then $\ov{TB(0,1)}\sps B(y,r)$, where $y$ is not necessarily 0. By translation and rescaling, $\ov{TB(0,1)}\sps B(0,\de)$. 

$\ov{TB(0,2^{-j-1})}\sps B(0,\fr{\de}{2^{j+1}})$ for all $j$.

Step 2: $T(B(0,1))\sps B(0,\fr{\de}{2})$. 

Pick $y\in B(0,\de/2)$. For $j=0$, there is $||x_0||\le 1/2$ with $||Tx_0-y||\le \de/2$. 

Next there exists $||x_1||<1/4$ with $||Tx_1+Tx_0-y||\le\de/4$. 

Inductively we get a sequence $||Tx_n+\cd+Tx_0-y||\le\fr{\de}{2^{n+1}}$. 

$||x_n||\le 2^{-(n+1)}$, $x_0+\cd+x_n\ra x$. $||x||<1$ $Tx=y$. 
%%%%
\end{proof}

This is equivalent to the next theorem.

\begin{thm}[Closed Graph]
Let $X,Y$ Banach and $T:X\ra Y$ linear. Define $G_T\sbs X\ti Y$ by $G_T=\{(x,Tx): x\in X\}$ Assume $G_T$ is closed. Then $T$ is continuous. 
\end{thm}
(Note: It's an if and only if. The other direction is trivial.) 
\begin{proof}
Note that $X\ti Y$ is Banach. $||(x,y)||=||x||_X+||y||_Y$. 

$G_T$ is a closed linear subspace of a Banach space. Hence $G_T$ is Banach. 

We have two maps. $\pi_1:X\ti Y\ra X$ by $\pi_1(x,y)=x$. Then $\pi_2:X\ti Y\ra Y$ by $\pi_2(x,y)=y$. These are both linear and continuous. 

$\pi_1|_{G_T}:G_T\ra X$ is linear and continuous, 1-1, and onto. (Note, it's 1-1 because $T$ is a function.) Then $\pi_1^{-1}:X\ra G_T$ exists and is continuous by open mapping theorem. 

Now $T=\pi_2\pi_1^{-1}$ is composition of continuous and hence is continuous. 
\end{proof}

\subsection{Application}

Let $T$ be the torus. $(P_nf)(x)=\fr{1}{2\pi}\int_0^{2\pi}D_n(\al-\be)f(\be)d\be$. 

$C(T)\sps L^2(T)$. Here we mean that the inclusion map is continuous. 

$i(f)=f$, $i:C(T)\ra L^2(T)$. We want to show $i\in L(C(t),L^2(T))$. We need to show $||i||<\infty$, i.e.\ there exists $C>0$ such that for any $f\in C(T)$, $||i(f)||_{L^2(T)}\le C||f||_{C(T)}$. 

\[||i(f)||_{L^2}=\left(\fr{1}{2\pi}\int_0^{2\pi}|f|^2d\al\right)^{1/2}?\le? C\sup_{\al\in[0,2\pi]}|f(\al)|=C||f||_{C(T)}\]

$C=1$ (???)

For all $\al$, $\sup_n|P_nf(\al)|$. . .

For all $\al$ there exists dense $G_{\de}$ in $C(T)$ so that for all $f\in G_{\de}$, $\sup_{n}|(P_nf)(\al)|=\infty$. 


\end{document}